% PADL'09
\pdfoutput=1
\documentclass[]{INCLUDES/llncs}
\usepackage{dcpic,pictexwd}
\input TOOLS/hheader.tex

\begin{document}

\title{
  %Declarative Combinatorics:
  Ranking and Unranking Functions for BDDs and MTBDDs with Applications to
  Circuit Minimization
  % and Random Test Generation
  }

\author{Paul Tarau\inst{1} \and Brenda Luderman\inst{2}}
\institute{
   {Department of Computer Science and Engineering}\\
   {University of North Texas}\\
   {Denton, Texas}\\
   {\em tarau@cs.unt.edu}\\
\and
   {ACES CAD}\\
   {Lewisville, Texas, USA}\\
   {\em brenda.luderman@gmail.com}\\
}
\maketitle

\date{}

\begin{abstract}
We describe Haskell implementations of 
combinatorial generation algorithms
with focus on boolean functions and
logic circuit representations.
Using {\em pairing} and {\em unpairing} functions on natural
number representations of truth tables, we derive 
an encoding for Binary Decision Diagrams (BDDs) with the unique property 
that its boolean evaluation faithfully mimics
its structural conversion to a natural number
through recursive application of a matching pairing function.
We then use this result to derive {\em ranking} and {\em unranking}
functions for BDDs and reduced BDDs.
Finally, a generalization of the encoding 
techniques to Multi-Terminal BDDs and an application 
to BDD minimization are described.

The paper is organized as a self-contained literate Haskell program,
available at \url{http://logic.csci.unt.edu/tarau/research/2008/fBDD.zip}.

{\em 
{\bf Keywords:}
binary decision diagrams,
encodings of boolean functions,
pairing/unpairing functions,
ranking/unranking functions for BDDs and MTBDDs,
computational mathematics in Haskell
}
\end{abstract}

\section{Introduction}

This paper is an exploration with functional programming tools of {\em ranking}
and {\em unranking} problems on Binary Decision Diagrams. The paper is part
of a larger effort to cover in a declarative programming 
paradigm some fundamental combinatorial generation and boolean function
manipulation algorithms along the lines of Knuth's recent
work \cite{knuth06draft}.

The paper is organized as follows:

Sections \ref{bits} and \ref{bdds} overview efficient evaluation of boolean
formulae in Haskell using bitvectors represented as arbitrary length integers
and Binary Decision Diagrams (BDDs).
Section \ref{pairings} introduces pairing/unpairing
functions acting directly on bitlists.
Section \ref{encbdd} introduces a novel BDD encoding (based on our unpairing
functions) and discusses the surprising equivalence between boolean evaluation of BDDs
and the inverse of our encoding.
Section \ref{rank} describes {\em ranking} and {\em unranking}
functions for BDDs and reduced BDDs.
Section \ref{gen} extends our techniques to BDDs with arbitrary variable
order and Multi-Terminal BDDs and
applications to BDD-based circuit minimization and generation of random test
data.
Sections \ref{related} and \ref{concl}
discuss related work, future work and conclusions.

\begin{comment}
\begin{code}
module BDD where
import Data.List
import Data.Bits
import Data.Array
import Random
\end{code}
\end{comment}

\section{Evaluation of Boolean 
Functions with Bitvector Operations}\label{bits}

Evaluation of a boolean function can be performed one 
bit at a time as in the function {\tt if\_then\_else}
\begin{code}
if_then_else 0 _ z = z
if_then_else 1 y _ = y
\end{code}
Unfortunately this does not take advantage of the ability of modern hardware to
perform such operations one word a time - with the instant benefit of a
speed-up proportional to the word size.
An alternate representation, adapted
from \cite{knuth06draft} uses integer encodings 
of $2^n$ bits for each boolean variable $x_0,\ldots,x_{n-1}$. 
Bitvector operations are used to evaluate all
value combinations at once.

\begin{prop}
Let $x_k$ be a variable for $0 \leq k<n$
where $n$ is the number of distinct variables in a 
boolean expression. Then column $k$ of the truth table
represents, as a bitstring, the natural number:

\begin{equation} \label{var}
x_k={(2^{2^n}-1)}/{(2^{2^{n-k-1}}+1)} 
\end{equation}
\end{prop}

\noindent For instance, if $n=2$, the formula computes 
$x_0=3=[0,0,1,1]$ and $x_1=5=[0,1,0,1]$.

The following functions, working with arbitrary length bitstrings are
used to evaluate the [0..n-1] variables $x_k$ with formula \ref{var} and map the
constant {\tt 1} to the bitstring of length $2^n$, {\tt 111..1}:
\begin{code}
var_n n k = var_mn (bigone n) n k
var_mn mask n k = mask `div` (2^(2^(n-k-1))+1)
bigone nvars = 2^2^nvars - 1
\end{code}
Variables representing such bitstring-truth tables 
(seen as {\em projection functions}) 
can be combined with the usual bitwise integer operators, 
to obtain new bitstring truth tables, 
encoding all possible value combinations of their arguments.
Note that the constant $0$ is represented as $0$ while the constant $1$
is represented as $2^{2^n}-1$, corresponding to a column in
the truth table containing ones exclusively.

\section{Binary Decision Diagrams} \label {bdds}

We have seen that Natural Numbers in $[0..{2^{2^n}-1}]$ can be used as
representations of truth tables defining $n$-variable boolean functions.
A binary decision diagram (BDD) \cite{bryant86graphbased} is an ordered binary 
tree obtained from a boolean function, by assigning its variables, one at a time, 
to {\tt 0}  (right branch) and {\tt 1} (left branch). 

The construction is known as Shannon expansion \cite{shannon_all}, and is
expressed as a decomposition of a function in two {\em cofactors}, $f[x
\leftarrow 0]$ and $f[x \leftarrow 1]$

\begin{equation}
f(x)= (\bar{x} \wedge f[x \leftarrow 0]) \vee (x \wedge f[x \leftarrow 1])
\end{equation}

\noindent where $f[x \leftarrow a]$ is computed 
by uniformly substituting $a$ for $x$ in $f$. Note that by using the more
familiar boolean {\tt if-the-else} function, the Shannon expansion can also
 be expressed as:

\begin{equation}
f(x) = if~x~then~f[x \leftarrow 1]~else~f[x \leftarrow 0]
\end{equation}

Alternatively, we observe that the Shannon expansion
can be directly derived from a $2^n$ size truth table, 
using bitstring operations on encodings of its $n$ variables.
Assuming that the first column of a truth table corresponds to 
variable $x$, $x=0$ and $x=1$ mask out, respectively, 
the upper and lower half of the truth table.

{\em Seen as an operation on bitvectors, the Shannon expansion 
(for a fixed number of variables) defines a bijection associating 
a pair of natural numbers (the cofactor's truth tables) 
to a natural number (the function's truth table), 
i.e. it works as a {\em unpairing function}.}

\section{Pairing/Unpairing} \label{pairings}

Let $Nat$ denote the set of natural numbers (0 included). A {\em pairing
function} is an isomorphism $f:Nat \times Nat \rightarrow Nat$. 
Its inverse is called an {\em unpairing function}.

We introduce here an unusually simple pairing function 
(also mentioned in \cite{pigeon}, p.142).
The function {\tt bitpair} works by splitting a 
number's big endian bitstring
representation into odd and even bits, 
while its inverse {\tt bitunpair}
blends the odd and even bits back together.

\begin{code}
type Nat = Integer
data Nat2 = P Nat Nat deriving (Eq,Ord,Read,Show)

bitpair ::  Nat2 -> Nat
bitpair (P i j) = 
  set2nat ((evens i) ++ (odds j)) where
    evens x = map (2*) (nat2set x)
    odds y = map succ (evens y)

bitunpair :: Nat->Nat2  
bitunpair n = P (f xs) (f ys) where 
  (xs,ys) = partition even (nat2set n)
  f = set2nat . (map (`div` 2))
\end{code}
The functions {\tt set2nat} and {\tt nat2set} convert to/from
natural numbers to lists of exponents of 2 representing positions of
bits=1.
\begin{code}  
nat2set n | n>=0 = nat2exps n 0 where
  nat2exps 0 _ = []
  nat2exps n x = 
    if (even n) then xs else (x:xs) where
      xs=nat2exps (n `div` 2) (succ x)

set2nat ns = sum (map (2^) ns)  
\end{code}

The transformation of the bitlists
is shown in the following example 
with bitstrings aligned:
\begin{codex}
*BDD> bitunpair 2008
  (60,26)

-- 2008:[0, 0, 0, 1, 1, 0, 1, 1, 1, 1, 1]
--   60:[      0,    1,    1,    1,    1]
--   26:[   0,    1,    0,    1,    1   ]
\end{codex}

\section{Pairing Functions and Encodings of Binary Decision Diagrams}
\label{encbdd}

We show in this section that a Binary Decision Diagram ($BDD$)
representing the same logic function as an $n$-variable $2^n$ bit truth
table can be obtained by applying {\tt bitunpair} recursively to {\tt tt}.
More precisely, we show that applying this {\em unfolding} operation
results in a complete binary tree of depth $n$ representing
a $BDD$ that returns {\tt tt} when evaluated applying
its boolean operations.

The binary tree type {\tt BT} has the constants {\tt
B0} and {\tt B1} as leaves representing the boolean values $0$ and $1$.
Internal nodes (that represent {\tt if-then-else} decision points), 
are marked with the constructor {\tt D}. 
We also add integers, representing logic
variables, ordered identically in each
branch, as first arguments of {\tt D}. 
The two other arguments are subtrees 
that represent {\tt THEN} 
and {\tt ELSE} branches:
\begin{code}
data BT a = B0 | B1 | D a (BT a) (BT a) deriving (Eq,Ord,Read,Show)
\end{code}

The constructor {\tt BDD} wraps together a 
tree of type {\tt BT} and the number of logic
variables occurring in it.
\begin{code}
data BDD a = BDD a (BT a) deriving (Eq,Ord,Read,Show)
\end{code}

\subsection{Unfolding natural numbers to binary trees with {\tt bitunpair}}
The following functions apply {\tt bitunpair} recursively, 
on a Natural Number {\tt tt}, 
seen as an $n$-variable $2^n$ bit truth table, 
to build a complete binary tree of depth $n$, 
that we represent using the {\tt BDD} data type. 
\begin{code}
unfold_bdd :: Nat2 -> BDD Nat
unfold_bdd (P n tt) = BDD n bt where 
  bt=if tt<max then shf bitunpair n tt
     else error 
       ("unfold_bdd: last arg "++ (show tt)++
       " should be < " ++ (show max))
     where max = 2^2^n

  shf _ n 0 | n<1 =  B0
  shf _ n 1 | n<1 =  B1
  shf f n tt = D k (shf f k tt1) (shf f k tt2) where
    k=pred n
    P tt1 tt2=f tt
\end{code}
The examples below
show results returned by {\tt unfold\_bdd} 
for the $2^{2^n}$ truth tables associated to $n$ variables,
for $n=2$:
\begin{codex}
 BDD 2 (D 1 (D 0 B0 B0) (D 0 B0 B0))
 BDD 2 (D 1 (D 0 B1 B0) (D 0 B0 B0))
 BDD 2 (D 1 (D 0 B0 B0) (D 0 B1 B0))
 ...
 BDD 2 (D 1 (D 0 B1 B1) (D 0 B1 B1))
\end{codex}
Note that no boolean operations have been performed so far
and that we still have to prove that such
trees actually represent BDDs associated to truth tables.

\subsection{Folding binary trees to natural numbers with {\tt bitpair}}
One can ``evaluate back'' the binary tree of data type BDD,
by using the pairing function {\tt bitpair}.  
The inverse of {\tt unfold\_bdd} is implemented as follows:
\begin{code}
fold_bdd :: BDD Nat -> Nat2
fold_bdd (BDD n bt) = 
  P n (rshf bitpair bt) where
    rshf rf B0 = 0
    rshf rf B1 = 1
    rshf rf (D _ l r) = 
      rf (P (rshf rf l) (rshf rf r))
\end{code}
Note that this is a purely structural operation
and that integers in first argument position
of the constructor {\tt D} are actually ignored.

The two bijections, inverses of each other, work as follows:
\begin{codex}
*BDD>unfold_bdd (P 3 42)
  BDD 3 
    (D 2 
      (D 1 (D 0 B0 B0) 
           (D 0 B0 B0)) 
      (D 1 (D 0 B1 B1) 
           (D 0 B1 B0)))

*BDD>fold_bdd it
  42
\end{codex}

\subsection{Boolean Evaluation of BDDs}
Practical uses of BDDs involve reducing them by
sharing nodes and eliminating identical
branches \cite{bryant86graphbased}.
Note that in this case {\tt bdd2nat} 
might give a different result as it computes
different pairing operations. 
Fortunately,
we can try to fold the binary tree 
back to a natural number
by evaluating it as a boolean
function.

The function {\tt eval\_bdd} describes the $BDD$ evaluator:
\begin{code}
eval_bdd (BDD n bt) = eval_with_mask (bigone n) n bt
 
eval_with_mask m _ B0 = 0 
eval_with_mask m _ B1 = m
eval_with_mask m n (D x l r) = 
  ite_ (var_mn m n x) 
         (eval_with_mask m n l) 
         (eval_with_mask m n r) 
\end{code}

The {\em projection functions} {\tt var\_mn} defined in section \ref{bits}
can be combined with the usual bitwise integer operators, 
to obtain new bitstring truth tables, 
encoding all possible value combinations of their arguments.
Note that the constant $0$ evaluates to $0$ while the constant $1$
is evaluated as $2^{2^n}-1$ by the function {\tt bigone}.

The function {\tt ite\_} used in {\tt eval\_with\_mask} 
implements the boolean function  {\tt if x then t else e}
using arbitrary length bitvector operations:
\begin{code}
ite_ x t e = ((t `xor` e).&.x) `xor` e
\end{code}

\noindent {\em 
As the following example shows, it turns out that
boolean evaluation {\tt eval\_bdd}
faithfully emulates {\tt fold\_bdd}!
}

\begin{codex}
*BDD> unfold_bdd (P 3 42)
BDD 3 (D 2 (D 1 (D 0 B0 B0) (D 0 B0 B0)) 
           (D 1 (D 0 B1 B1) (D 0 B1 B0)))
*BDD> eval_bdd it
42
\end{codex}

\subsection{The Equivalence}
We now state the surprising (and new!) result
that boolean evaluation and structural transformation with
repeated application of
{\em pairing}
produce the same result:

\begin{prop} \label{tt}
The complete binary tree of depth $n$, obtained by recursive 
applications of {\tt bitunpair} on a truth table $tt$
computes an (unreduced) BDD, that, when evaluated, 
returns the truth table, i.e.
\begin{equation}
fold\_bdd \circ unfold\_bdd \equiv id
\end{equation}

\begin{equation}
eval\_bdd \circ unfold\_bdd \equiv id
\end{equation}
\end{prop}
\begin{proof} The function {\tt unfold\_bdd} builds a
binary tree by splitting the bitstring $tt \in [0..2^n-1]$ up to depth $n$. 
Observe that this corresponds to the Shannon expansion \cite{shannon_all} of the
formula associated to the truth table, using variable order $[n-1,...,0]$.
Observe that the effect of {\tt bitunpair} is the same as
\begin{itemize}
  \item the effect of {\tt var\_mn m n (n-1)} 
     acting as a mask selecting the left branch, and
\item 
     the effect of its complement, acting as a mask selecting the right
     branch.
\end{itemize}
Given that $2^n$ is the double of $2^{n-1}$, the same invariant holds at each
step, as the bitstring length of the truth table reduces to half. 
\end{proof}
We can thus assume from now on, that the BDD data type defined in
section \ref{encbdd} actually represents BDDs mapped one-to-one to truth tables
given as natural numbers.

\section{Ranking and Unranking of BDDs} \label{rank}

One more step is needed to extend the mapping between $BDDs$ with $n$
variables to a bijective mapping from/to $Nat$: 
we will have to ``shift towards infinity'' 
the starting point of each new block\footnote{defined by the same number of
variables} of BDDs in $Nat$ as BDDs of larger and larger sizes are enumerated.

First, we need to know by how much - so we count the number
of boolean functions with up to $n$ variables.
\begin{code}
bsum 0 = 0
bsum n | n>0 = bsum1 (n-1)

bsum1 0 = 2
bsum1 n | n>0 = bsum1 (n-1)+ 2^2^n
\end{code}
The stream of all such sums can now be generated as usual\footnote{{\tt bsums}
is sequence A060803 in The On-Line Encyclopedia of Integer
Sequences, \url{http://www.research.att.com/~njas/sequences}}:
\begin{code}
bsums = map bsum [0..]
\end{code}
\begin{codex}
*BDD> genericTake 7 bsums
  [0,2,6,22,278,65814,4295033110]
\end{codex}

What we are really interested in, is decomposing {\tt n} into
the distance {\tt n-m} to the
last {\tt bsum} {\tt m} smaller than {\tt n},
and the index that generates the sum, {\tt k}.
\begin{code}
to_bsum n = (k,n-m) where 
  k=pred (head [x|x<-[0..],bsum x>n])
  m=bsum k
\end{code}
{\em Unranking} of an arbitrary BDD is now easy - the index {\tt k}
determines the number of variables and {\tt n-m} determines
the rank. Together they select the right BDD
with {\tt unfold\_bdd} and {\tt bdd}.
\begin{code}
nat2bdd n = unfold_bdd (P k n_m)
  where (k,n_m)=to_bsum n
\end{code}
{\em Ranking} of a BDD is even easier: we shift its rank
within the set of BDDs with {\tt nv} 
variables, by the value {\tt (bsum nv)} that
counts the ranks previously assigned.
\begin{code}
bdd2nat bdd@(BDD nv _) =  ((bsum nv)+tt) where
  P _ tt =fold_bdd bdd
\end{code}
As the following example shows,
{\tt bdd2nat}
implements the inverse of
{\tt nat2bdd}.
\begin{codex}
*BDD> nat2bdd 42
BDD 3 (D 2 (D 1 (D 0 B0 B1) (D 0 B1 B0)) 
           (D 1 (D 0 B0 B0) (D 0 B0 B0)))
*BDD> bdd2nat it
42
\end{codex}

We can now repeat the {\em ranking} function construction for {\tt eval\_bdd}:
\begin{code}
ev_bdd2nat bdd@(BDD nv _) = (bsum nv)+(eval_bdd bdd)
\end{code}
We can confirm that {\tt ev\_bdd2nat} also acts as an inverse to
{\tt nat2bdd}:
\begin{codex}
*BDD> ev_bdd2nat (nat2bdd 2008)
2008
\end{codex}

\subsection{Reducing the $BDDs$}
We sketch here a simplified reduction mechanism for BDDs
eliminating identical branches. Note that the general
mechanism involves DAGs and provides also node sharing
\cite{bryant86graphbased}.

The function {\tt bdd\_reduce} reduces a $BDD$ by collapsing identical 
left and right subtrees, and the function {\tt bdd} 
associates this reduced form to $n \in Nat$.
\begin{code}
bdd_reduce (BDD n bt) = BDD n (reduce bt) where
  reduce B0 = B0
  reduce B1 = B1
  reduce (D _ l r) | l == r = reduce l
  reduce (D v l r) = D v (reduce l) (reduce r)

unfold_rbdd = bdd_reduce . unfold_bdd  
\end{code}

The results returned by {\tt unfold\_rbdd} for {\tt n=2} are:
\begin{codex}
  BDD 2 (C 0)
  BDD 2 (D 1 (D 0 (C 1) (C 0)) (C 0))
  BDD 2 (D 1 (C 0) (D 0 (C 1) (C 0)))
  BDD 2 (D 0 (C 1) (C 0))
  ...
  BDD 2 (D 1 (D 0 (C 0) (C 1)) (C 1))
  BDD 2 (C 1)
\end{codex}
We can now define the {\em unranking} operation on reduced BDDs
\begin{code}
nat2rbdd = bdd_reduce . nat2bdd 
\end{code}

To be able to compare its space complexity
with other representations we define 
a size operation on a BDD as follows:
\begin{code}
bdd_size (BDD _ t) = 1+(size t) where
  size B0 = 1
  size B1 = 1
  size (D _ l r) = 1+(size l)+(size r)
\end{code}

\section{Generalizing BDD ranking/unrankig functions} \label{gen}

\subsection{Encoding BDDs with Arbitrary Variable Order}
While the encoding built around the equivalence described in Prop. \ref{tt}
between bitwise pairing/unpairing operations and boolean decomposition
is arguably as simple and elegant as possible, it is useful
to parametrize BDD generation with respect to an arbitrary
variable order. This is of particular importance when using
BDDs for circuit minimization, as different variable orders
can make circuit sizes flip from linear to exponential in
the number of variables \cite{bryant86graphbased}.

Given a permutation of $n$ variables represented as
natural numbers in $[0..n-1]$ and a truth table
$tt \in [0..2^{2^n}-1]$ we can define: 
\begin{code}
to_bdd vs tt | 0<=tt && tt <= m = 
  BDD n (to_bdd_mn vs tt m n) where
    n=genericLength vs
    m=bigone n
to_bdd _ tt = error 
   ("bad arg in to_bdd=>" ++ (show tt)) 
\end{code}
where the function {\tt to\_bdd\_mn} recurses over
the list of variables {\tt vs} and applies
Shannon expansion \cite{shannon_all},
expressed as bitvector operations. This computes the cofactor functions
$f1$ and $f0$, to be used as {\tt then} and {\tt else}
branches, when evaluating back the BDD to a truth table
with if-the-else functions.
\begin{code}
to_bdd_mn []      0 _ _ = B0
to_bdd_mn []      _ _ _ = B1
to_bdd_mn (v:vs) tt m n = D v l r where
  cond=var_mn m n v
  f0= (m `xor` cond) .&. tt
  f1= cond .&. tt 
  l=to_bdd_mn vs f1 m n
  r=to_bdd_mn vs f0 m n
\end{code}
\begin{prop}
The function {\tt to\_bdd} builds an (unreduced) BDD corresponding
to a truth table {\tt tt} for variable order {\tt vs} that returns
{\tt tt}, when evaluated as a boolean function.
\end{prop}
We can reduce the resulting BDDs, and convert back from BDDs and reduced BDDs to
truth tables with boolean evaluation: 
\begin{code}
to_rbdd vs tt = bdd_reduce (to_bdd vs tt)
from_bdd bdd = eval_bdd bdd
\end{code}

Finally, we can, obtain an optimal BDD expressing a logic function of $n$
variables given as a truth table as follows:
\begin{code}
search_bdd minORmax n tt = snd $ foldl1 minORmax 
 (map (sized_rbdd tt) (all_permutations n)) where
    sized_rbdd tt vs = (bdd_size b,b) where 
      b=to_rbdd vs tt
 
all_permutations n = permute [0..n-1] where

  permute [] = [[]]
  permute (x:xs) = [zs| ys<-permute xs, zs<-insert x ys]

  insert a [] = [[a]]
  insert a (x:xs) = (a:x:xs):[(x:ys) | ys<-(insert a xs)]
\end{code}
%$
The function {\tt search\_bdd min}
can be used for multilevel 
boolean formula minimization on functions with
up to 6-7 arguments.
\begin{codex}
*BDD> search_bdd min 3 42
BDD 3 (D 2 B0 (D 0 B1 (D 1 B1 B0)))
*BDD> search_bdd min 4 2008
BDD 4 (D 0 (D 3 (D 1 B0 B1) (D 2 B0 B1)) 
           (D 3 (D 1 B1 B0) (D 1 (D 2 B1 B0) B0)))
*BDD> search_bdd min 3 2008
BDD 7 (D 1 (D 2 (D 4 (D 3 (D 0 (D 5 ...
      ... (D 0 (D 5 B0 B1) B0)))))
*BDD> bdd_size it
110
\end{codex}

Let us consider the classic problem of synthesizing a half adder, composed
of an XOR (\verb~^~) and an AND  (\verb~*~) function.

It is interesting to see how the BDD minimization algorithm compares with
``perfect circuits'' provided by an exact synthesizer 
as the one described in \cite{iclp07,cf08}.
The output of the exact synthesizer uses a graph representation where
the 4-th argument names the output connection: 
\begin{codex}
?- syn([ite],[0,1],[ite(A,B^C,B*C)]).
[TTs = [22],MG = 24]
syn(3,24,[ite],[0,1],[22]).
[0,0,0]:0
[0,0,1]:0
[0,1,0]:0
[0,1,1]:1
[1,0,0]:0
[1,0,1]:1
[1,1,0]:1
[1,1,1]:0

[A,B,C]:
   [ite(A,0,1,D),
    ite(D,0,B,E),
    ite(B,D,A,F),
    ite(C,F,E,G)] = [G]:[22].
\end{codex}
Note that 4 {\tt ITE} gates (2-1 mux with 1-0 selection lines) are used
to combine the {\tt XOR} and {\tt AND} functions into a single output
function, with the upper half of the truth table representing
the {\tt AND} and the lower half representing the {\tt XOR}.

When running {\tt to\_min\_bdd} on the 3-variable function 22 representing as a
natural number the truth table of our our half adder, we obtain:
\begin{codex}
*BDD> search_bdd min 3 22
BDD 3 (D 0 (D 1 (D 2 B0 B1) 
                (D 2 B1 B0)) 
           (D 1 (D 2 B1 B0) 
                 B0))
\end{codex}
Assuming the sharing of the two {\tt (D 2 B1 B0)} nodes we
can see that while we do not obtain the exact minimum of 4 in this case,
we get close enough (5 gates). The diagrams  in Fig. \ref{ite} show the
actual circuits, with variables {\tt 0,1,2} renamed as {\tt A,B,C} for easier
comparison.

\HFIGS{ite}
{Binary decision diagrams associated to half adder 
$ITE(A \oplus B,A \wedge B)$} 
{As minimal BDDs (exact synthesis)} 
{As minimal ROBDDs}{ite4}{ite5}

\subsection{Multi-Terminal Binary Decision Diagrams (MTBDD)} \label{multi}
MTBDDs \cite{DBLP:journals/fmsd/FujitaMY97,CBGP08} are a natural generalization
of BDDs allowing non-binary values as leaves.
Such values are typically 
bitstrings representing the outputs
of a multi-terminal boolean function,
encoded as unsigned integers.

We shall now describe an encoding of $MTBDDs$
that can be extended to ranking/unranking functions,
in a way similar to $BDDs$ as shown in section \ref{rank}.

Our {\tt MTBDD} data type is a binary tree like the one used for $BDDs$,
parameterized by two integers {\tt m} and {\tt n}, indicating
that an MTBDD represents a function from $[0..n-1]$ to $[0..m-1]$,
or equivalently, an $n$-input/$m$-output boolean function.

\begin{code}   
data MT a = L a | M a (MT a) (MT a) 
             deriving (Eq,Ord,Read,Show)
data MTBDD a = MTBDD a a (MT a) deriving (Show,Eq)
\end{code}

The function  {\tt to\_mtbdd} creates,
from a natural number tt representing a truth table,
an MTBDD representing
functions of type $N \rightarrow M$ with $M=[0..2^m-1], N=[0..2^n-1]$.
Similarly to a BDD, it is represented as binary tree 
of $n$ levels, except that its leaves are in $[0..{2^m}-1]$.
\begin{code}
to_mtbdd m n tt = MTBDD m n r where 
  mlimit=2^m
  nlimit=2^n
  ttlimit=mlimit^nlimit
  r=if tt<ttlimit 
    then (to_mtbdd_ mlimit n tt)
    else error 
      ("bt: last arg "++ (show tt)++
      " should be < " ++ (show ttlimit))
\end{code}
Given that correctness of the range of
{\tt tt} has been checked, the function {\tt to\_mtbdd\_} 
applies {\tt bitunpair} 
recursively up to depth $n$, where
leaves in range $[0..mlimit-1]$ are created.
\begin{code}  
to_mtbdd_ mlimit n tt|(n<1)&&(tt<mlimit) = L tt
to_mtbdd_ mlimit n tt = (M k l r) where 
   P x y=bitunpair tt
   k=pred n
   l=to_mtbdd_ mlimit k x
   r=to_mtbdd_ mlimit k y
\end{code}
Converting back from $MTBDDs$ to natural numbers is
basically the same thing as for $BDDs$, except that
assertions about the range of leaf data are enforced.
\begin{code}
from_mtbdd (MTBDD m n b) = from_mtbdd_ (2^m) n b

from_mtbdd_ mlimit n (L tt)|(n<1)&&(tt<mlimit)=tt
from_mtbdd_ mlimit n (M _ l r) = tt where 
   k=pred n
   x=from_mtbdd_ mlimit k l
   y=from_mtbdd_ mlimit k r
   tt=bitpair (P x y)
\end{code}
The following examples show that {\tt to\_mtbdd} and {\tt from\_mtbdd}
are indeed inverses values in $[0..2^n-1] \times [0..2^m-1]$. 
\begin{codex}
>to_mtbdd 3 3 2008
  MTBDD 3 3 
    (M 2 
      (M 1 
         (M 0 (L 2) (L 1)) 
         (M 0 (L 2) (L 1))) 
      (M 1 
         (M 0 (L 2) (L 0)) 
         (M 0 (L 1) (L 1))))

>from_mtbdd it
2008

>mprint (to_mtbdd 2 2) [0..3]
  MTBDD 2 2 
    (M 1 (M 0 (L 0) (L 0)) (M 0 (L 0) (L 0)))
  MTBDD 2 2 
    (M 1 (M 0 (L 1) (L 0)) (M 0 (L 0) (L 0)))
  MTBDD 2 2 
    (M 1 (M 0 (L 0) (L 0)) (M 0 (L 1) (L 0)))
  MTBDD 2 2 
    (M 1 (M 0 (L 1) (L 0)) (M 0 (L 1) (L 0)))
\end{codex}

\subsection{Generating Random BDDs and MTBDDs}
Random generation of BDDs and MTBDDs have practical uses
in testing and benchmarking of various electronic design
automation tools and methodologies.

Deriving mechanisms for uniform generation of random instances
is a classic application of ranking/unranking functions.
Given a one-to-one mapping to $Nat$ it reduces
to the simpler problem of uniform generation of
natural numbers.

After customizing Haskell's library random generator
\begin{code}
nrandom_nats smallest largest n seed= 
  genericTake n 
    (randomRs (smallest,largest) (mkStdGen seed))
\end{code}
one can define:
\begin{code}
nrandom converter smallest largest n seed =
  map converter (nrandom_nats smallest largest n seed)
\end{code}

To generate 3 small instances of reduced BDD mapped to
natural numbers from 10 to 20 one can write:
\begin{codex}
*BDD> nrandom nat2rbdd 10 20 3 77
[ BDD 2 (D 1 (D 0 B1 B0) B1),
  BDD 2 (D 1 (D 0 B0 B1) B1),
  BDD 2 (D 0 B0 B1)]
\end{codex}

To generate an instance of a random 3-in/3-out MTBDD mapped to
natural numbers from 1000 to 2000 one can write:
\begin{codex}  
*BDD> head $ nrandom (to_mtbdd 3 3) 1000 2000 1 1
MTBDD 3 3 (M 2 (M 1 (M 0 (L 2) (L 1)) (M 0 (L 2) (L 1))) 
          (M 1 (M 0 (L 0) (L 1)) (M 0 (L 0) (L 1))))  
\end{codex}
%$
One can see the average size reduction from
BDDs to reduced BDDs with something like:
\begin{codex}
*BDD> sum $ map bdd_size $ nrandom nat2bdd 1000 2000 10 7
320
*BDD> sum $ map bdd_size $ nrandom nat2rbdd 1000 2000 10 7
194
\end{codex}
Or, one can see the size reductions due to trying
all possible variable orders on random BDDs (in the case of 100 random 4 and 5
variable functions):
\begin{codex}
*BDD> sum $ map bdd_size (nrandom (search_bdd max 4) 0 (2^2^4-1) 100 77)
2384
*BDD> sum $ map bdd_size (nrandom (search_bdd min 4) 0 (2^2^4-1) 100 77)
1744
*BDD> sum $ map bdd_size (nrandom (search_bdd max 5) 0 (2^2^5-1) 100 77)
4812
*BDD> sum $ map bdd_size (nrandom (search_bdd min 5) 0 (2^2^5-1) 100 77)
3432
\end{codex}
Such results are useful in evaluating the pros/cons of various circuit
minimization strategies. 
Needless to say, one might notice the compactness and elegance of a
declarative language like Haskell for such ad-hoc tasks, 
recommending it as a powerful scripting language for 
electronic design automation tools.

\section{Related work} \label{related}
Pairing functions have been used for work on decision problems as early 
as \cite{robinson50}. 

BDDs are the dominant boolean function representation in
the field of circuit design automation
\cite{DBLP:journals/jcsc/MeinelT99}.

Besides their uses in circuit design automation,
MTBDDs have been used in model-checking and
verification of arithmetic circuits \cite{DBLP:journals/fmsd/FujitaMY97,CBGP08}.

BDDs have also been used in a Genetic Programming context
\cite{DBLP:conf/ices/SakanashiHIK96}
as a representation of evolving individuals subject to crossovers 
and mutations expressed as structural transformations.

\section{Conclusion and Future Work} \label{concl}
The surprising connection of bitstring based pairing/unpairing functions and
to BDDs came out as the indirect result of implementation
work on a number of practical applications.
Our initial interest has been triggered by applications of the 
encodings to combinational circuit synthesis \cite{iclp07,cf08}
and ongoing work on genetic programming algorithms.

We have found such encodings interesting as uniform 
building blocks for Genetic Programming applications.
In a Genetic Programming context \cite{koza92}, 
the bijections between bitvectors/natural numbers 
on one side, and trees/graphs representing BDDs on the other side, 
suggest exploring the mapping and its action on various
transformations as a phenotype-genotype connection. 
Given the connection between BDDs to
boolean and finite domain constraint solvers
it would be interesting to explore in that context,
efficient succinct data representations
derived from our BDD encodings.

%\newpage
%\bibliographystyle{plain}
%\bibliographystyle{plainnat}
\bibliographystyle{abbrv}
\bibliography{INCLUDES/theory,tarau,INCLUDES/proglang,INCLUDES/biblio,INCLUDES/syn}

\end{document}
