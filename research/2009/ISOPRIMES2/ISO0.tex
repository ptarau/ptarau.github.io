\begin{comment}
\pdfoutput=1
\documentclass[]{INCLUDES/llncs}
\usepackage{TOOLS/dcpic,pictexwd}
\input TOOLS/jhheader.tex
\title{
    Appendix
}
\author{}
\institute{}
\end{comment}

%\begin{document}
%\maketitle
%\date{}

\begin{comment}
In \cite{sac09fISO} an embedded
data transformation language is introduced, specified as a set of operations
on a groupoid of isomorphisms connecting various data types. To make the paper
self-contained as a literate Haskell program, we will provide in the
{\tt Appendix} the relevant subset of the commented code described in full
detail in \cite{sac09fISO}.
\end{comment}

\begin{comment}
\begin{code}
module ISO0 where
\end{code}
\end{comment}

\section*{Appendix}

\subsection*{An Embedded Data Transformation Language}

We will describe briefly the embedded data transformation language used in
this paper as a set of operations on a groupoid of isomorphisms. We will
then extended it with a set of higher order combinators mediating the
composition of the encodings and the transfer of operations between 
data types.

\subsubsection*{The Groupoid of Isomorphisms}

We implement an isomorphism between two objects X and Y as a 
Haskell data type encapsulating a bijection $f$ and its inverse $g$. 
We will call the {\em from} function the first component (a {\em
section} in category theory parlance) and
the {\em to} function the second component (a {\em retraction}) defining
the isomorphism.
We can organize isomorphisms as a {\em groupoid} as follows:

\begindc{\commdiag}[5]
\obj(14,11){$X$}
\obj(39,11){$Y$}
\mor(14,12)(39,12){$f=g^{-1}$}
\mor(39,10)(14,10){$g=f^{-1}$}
\enddc

\begin{code}
data Iso a b = Iso (a->b) (b->a)

from (Iso f _) = f
to (Iso _ g) = g

compose :: Iso a b -> Iso b c -> Iso a c
compose (Iso f g) (Iso f' g') = Iso (f' . f) (g . g')
itself = Iso id id
invert (Iso f g) = Iso g f
\end{code}
Assuming that for any pair of type {\tt Iso a b},  $f \circ g = id_a$ and $g
\circ f=id_b$, we can now formulate {\em laws} about these isomorphisms.
\begin{quote} 
The data type {\tt Iso} has a groupoid structure, i.e. the {\em compose}
operation, when defined, is associative, {\em itself} acts as an identity element
and {\em invert} computes the inverse of an isomorphism.
\end{quote}

\subsubsection*{Choosing a Root: Finite Sequences of Natural Numbers}
To avoid defining $n(n-1)/2$ isomorphisms between $n$ objects,
we choose a {\em Root} object to/from which we will actually
implement isomorphisms. We will extend our embedded
combinator language using the groupoid structure of the isomorphisms
to connect any two objects through isomorphisms to/from
the {\em Root}.
\begin{comment}
\begindc{\commdiag}[20]
\obj(10,10)[X]{$Root$}
\obj(14,10)[Aa]{$A$}

\obj(13,12)[Ac]{$B$}
\obj(12,13)[Ad]{$C$}

\obj(10,14)[Af]{$D$}

\obj(8,13)[Ah]{$E$}
\obj(7,12)[Ai]{$F$}

\obj(6,10)[Ak]{$G$}

\obj(7,8)[Am]{$H$}
\obj(8,7)[An]{$I$}

\obj(10,6)[Ap]{$J$}

\obj(12,7)[Ar]{$K$}
\obj(13,8)[As]{$L$}

\mor{X}{Aa}{$a$}

\mor{X}{Ac}{$b$}
\mor{X}{Ad}{$c$}

\mor{X}{Af}{$d$}

\mor{X}{Ah}{$e$}
\mor{X}{Ai}{$f$}

\mor{X}{Ak}{$g$}

\mor{X}{Am}{$h$}
\mor{X}{An}{$i$}

\mor{X}{Ap}{$j$}

\mor{X}{Ar}{$k$}
\mor{X}{As}{$l$}
\enddc
\end{comment}

Choosing a {\em Root} object is somewhat arbitrary, but it makes sense to
pick a representation that is relatively easy convertible to various
others and scalable to
accommodate large objects up to the runtime system's 
actual memory limits.

We will choose as our {\em Root} object {\em finite sequences of natural
numbers}. They can be seen as {\tt finite functions} from an initial 
segment of $Nat$, say $[0..n]$, to $Nat$.
We will represent them as lists i.e. their Haskell type is $[Nat]$.
\begin{code}
type Nat = Integer
type Root = [Nat]
\end{code}
We can now define an {\em Encoder} as an isomorphism
connecting an object to {\em Root} 
\begin{code}
type Encoder a = Iso a Root
\end{code}
together with the combinators {\em with} and {\em as}
providing an {\em embedded transformation language} for routing
isomorphisms through two {\em Encoders}.
\begin{code}  
with :: Encoder a->Encoder b->Iso a b
with this that = compose this (invert that)

as :: Encoder a -> Encoder b -> b -> a
as that this thing = to (with that this) thing
\end{code}
The combinator {\tt with} turns two Encoders
into an arbitrary isomorphism, i.e. acts as a connection hub between
their domains. The combinator {\tt as} adds a more convenient syntax
such that converters between A and B can be designed as:
\begin{codex}
a2b x = as A B x
b2a x = as B A x
\end{codex}
\vskip 0.30cm
\begindc{\commdiag}[5]
\obj(26,0){$Root$}
\obj(14,11){$A$}
\obj(39,11){$B$}

\mor(26,0)(39,10){$b$}
\mor(26,0)(14,10){$a^{-1}$}
\mor(39,10)(26,0){$b^{-1}$}
\mor(14,10)(26,0){$a$}
\mor(14,12)(39,12){$a2b=as~B~A$}
\mor(39,10)(14,10){$b2a=as~A~B$}
\enddc
\vskip 0.30cm
A particularly useful combinator that
transports binary operations from an Encoder to another, {\tt
borrow\_from}, can be defined as follows:
\begin{code}
borrow_from :: Encoder a -> (a -> a -> a) -> Encoder b -> b -> b -> b
borrow_from other op this x y = borrow2 (with other this) op x y
\end{code}
\noindent Given that $[Nat]$ has been chosen as the root, we will define our
finite function data type {\em fun} simply as the identity isomorphism 
on sequences in $[Nat]$.
\begin{code}  
fun :: Encoder [Nat]
fun = itself
\end{code}
%\end{document}
