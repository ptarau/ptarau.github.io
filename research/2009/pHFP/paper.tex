% ICLP'09 submission 82
\pdfoutput=1
\documentclass[]{INCLUDES/llncs}
\input TOOLS/pheader09.tex

\begin{document}

\title{
  Declarative Combinatorics: Ranking and Unranking of Hereditarily Finite
  Permutations }

\author{Paul Tarau}
\institute{
   Department of Computer Science and Engineering\\
   University of North Texas\\
   %P.O. Box 311366\\
   %Denton, Texas 76203\\
   {\em E-mail: tarau@cs.unt.edu}
}
\maketitle
\date{}

\begin{abstract}
Sets represent ``content'' in a pure way - order is immaterial. Permutations
represent ``order'' in a pure way - what is actually ordered is immaterial.
The paper will show that a similar ``fractal'' structure is shared when natural
number encodings of both sets and permutations are expanded recursively.

Starting from encodings for finite permutations 
based on Lehmer codes and factoradics, we derive through a 
process similar to Ackermann's encoding of
hereditarily finite sets, an encoding of 
{\em hereditarily finite permutations}.

The paper is organized as a self-contained literate Prolog program  available 
at \url{http://logic.cse.unt.edu/tarau/research/2009/pHFP.zip}.

{\em Keywords:}
logic programming and computational mathematics,
hereditarily finite permutations, 
encodings of permutations, factoradics
ranking/unranking bijections,
Ackermann's encoding of hereditarily finite sets

\end{abstract}

\section{Introduction}

This paper is an exploration with logic programming tools of {\em ranking} and
{\em unranking} problems on finite permutations and their related
hereditarily finite universe. The practical expressiveness of logic programming
languages (in particular Prolog) are put at test in the process. The paper is part
of a larger effort to cover in a declarative programming 
paradigm, arguably more elegantly, some fundamental combinatorial generation 
algorithms along the lines of \cite{knuth06draft}.

The paper is organized
as follows: section \ref{unrank} introduces generic 
ranking/unranking functions, section \ref{ack} introduces
 Ackermann's encoding in the more general case 
when {\em urelements} are present.
Ranking/unranking of permutations and Hereditarily Finite
Permutations as well as Lehmer codes and factoradics are covered in 
section \ref{perm}. Sections \ref{related} and \ref{concl}
discuss related work, future work and conclusions.
 
We will assume that the underlying
Prolog system supports the usual higher order 
function-style predicates {\tt
call/N}, {\tt findall/3}, {\tt maplist/N}, {\tt sumlist/2} or 
their semantic equivalents 
and a few well known library predicates, used mostly for list processing and
arithmetics. Arbitrary length integers are needed for 
some of the larger examples but 
their absence does not affect the correctness of
the code within the integer range provided by a given Prolog implementation.
Otherwise, the code in the paper, embedded in a literate programming LaTeX
file, is self contained and runs under {\em SWI-Prolog}. Note also that a few
utility predicates, not needed for following the main ideas of the paper,
are left out from the narrative and provided in the Appendix.

The paper is organized as follows: section \ref{unrank} introduces a general
ranking/unranking framework for multiway tree data types with atoms/urelements
and section \ref{ack} specializes it to hereditarily finite sets with
urelements. An encoding of finite permutations is given in section \ref{perm}
followed by the introduction of hereditarily finite permutations in section
\ref{hfp}, the crux of the paper.
Related work is discussed in section \ref{related}, followed by conclusions in
section \ref{concl}.

\section{Generic unranking and ranking with higher order predicates}
\label{unrank}
We will use, through the paper, a generic {\em multiway tree} type
distinguishing between atoms represented as (arbitrary length) integers and
sub-forests represented as Prolog lists.
Atoms will be
mapped to natural numbers in {\tt [0..Ulimit-1]}. 
Assuming that {\tt Ulimit} is fixed, we denote $A$ the set {\tt [0..Ulimit-1]}.
We denote $Nat$ the set of natural numbers and $T$ 
the set of trees of type $T$ with atoms in $A$.
\begin{df}
A ranking function on $T$ is a bijection $T \rightarrow Nat$.
An unranking function is a bijection $Nat \rightarrow T$.
\end{df}
{\em Ranking} functions can be traced back to 
G\"{o}del numberings \cite{Goedel:31,conf/icalp/HartmanisB74} 
associated to formulae. However, G\"{o}del numberings are typically only
injective functions, as their use in the proofs of G\"{o}del's incompleteness
theorems only requires injective mappings from well-formed formulae to
numbers. Together with their inverse {\em unranking} functions
they are also used in combinatorial and uniform random instance generation
\cite{conf/mfcs/MartinezM03,knuth06draft} algorithms.

\subsection{Unranking} As an adaptation of the {\em unfold} 
operation \cite{DBLP:journals/jfp/Hutton99,DBLP:conf/fpca/MeijerH95}, 
elements of $T$ will be mapped to natural numbers with a generic 
higher order function {\tt unrank\_} parameterized by the the natural number
{\tt Ulimit} and the transformer function {\tt F}:

\begin{code}
unrank_(Ulimit,_,N,R):-N>=0,N<Ulimit,!,R=N.
unrank_(Ulimit,F,N,R):-N>=Ulimit,
  N0 is N-Ulimit,
  call(F,N0,Ns),
  maplist(unrank_(Ulimit,F),Ns,R).
\end{code}
 A global constant provided by the predicate {\tt default\_ulimit},
will be used
through the paper to fix the default range of atoms as
 well as a default {\tt unrank} function:
Note also that we will use a syntactically more convenient {\tt DCG} 
notation for functional style
predicates, composed by chaining their arguments automatically with 
Prolog's {\tt DCG} transformation:
\begin{code}
unrank(F)-->
  default_ulimit(Ulimit),
  unrank_(Ulimit,F).

default_ulimit(L)-->{clause(ulimit(L),_)},!.
default_ulimit(0)-->[].
\end{code}
Note also that {\tt default\_ulimit} provides a default global value for
the number of atoms that can be customized with a dynamic clause {\tt ulimit/1}
if needed.

\subsection{Ranking} Similarly, as an adaptation of {\em fold}, generic inverse 
mappings {\tt rank\_(Ulimit,G)} and {\tt rank} from $T$ to $Nat$ are defined
as:
\begin{code}
rank_(Ulimit,_,N,R):-integer(N),N>=0,N<Ulimit,!,R=N.
rank_(Ulimit,G,Ts,R):-
  maplist(rank_(Ulimit,G),Ts,T),
  call(G,T,R0),
  R is R0+Ulimit.

rank(G)-->
  default_ulimit(Ulimit),
  rank_(Ulimit,G).  
\end{code}
Note that the guard in the second definition simply states 
correctness constraints ensuring 
that atoms belong to the same set $A$ for {\tt rank\_} and {\tt unrank\_}. 
This ensures that the following holds:

\begin{prop}
If the transformer function $F:Nat \rightarrow [Nat]$ is a bijection with
inverse G, such that $n \geq ulimit \wedge F(n)=[n_0,...n_i,...n_k]
\Rightarrow n_i<n$, then {\tt unrank} is a bijection from $Nat$ to $T$, 
with inverse {\tt rank} and the recursive computations 
of both functions terminate in a finite number of steps.

\noindent {\em Proof:} by induction on the structure of $Nat$ and $T$, using the
fact that {\tt maplist} preserves bijections.
\end{prop}

\section{Hereditarily finite sets and Ackermann's encoding} \label{ack}

The universe of hereditarily finite sets is best known as a model of the 
Zermelo-Fraenkel set theory with the axiom of infinity  replaced by its 
negation \cite{finitemath,DBLP:journals/jct/MeirMM83}. In a logic programming
context, it has been used for reasoning with sets, set constraints,
hypersets and bisimulations \cite{dovier00comparing,DBLP:journals/tplp/PiazzaP04}.

The universe of hereditarily finite sets is built from the empty set 
(or a set of {\em Urelements}) 
by successively applying powerset and set union operations.

Ackermann's encoding \cite{ackencoding,abian78,kaye07,arxiv:fISO} is a bijection
that maps hereditarily finite sets ($HFS$) to natural numbers ($Nat$) as follows:

\vskip 0.5cm
$f(x)$ = {\tt if} $x=\{\}$ {\tt then} $0$ {\tt else} $\sum_{a \in x}2^{f(a)}$
\vskip 0.5cm

Assuming $HFS$ extended with {\em Urelements} (atomic objects not having any
elements) a generic tree representation can be used for hereditarily finite
sets with urelements.

Ackermann's encoding can be seen as the recursive application of 
a bijection {\tt set2nat} from finite subsets of $Nat$ to $Nat$, that associates to 
a set of (distinct!) natural numbers a (unique!) natural number
\cite{arxiv:fISO}.
\begin{code} 
set2nat(Xs,N):-set2nat(Xs,0,N).

set2nat([],R,R).
set2nat([X|Xs],R1,Rn):-R2 is R1+(1<<X),set2nat(Xs,R2,Rn).
\end{code}
With this representation, Ackermann's encoding from $HFS$ to $Nat$ {\tt hfs2nat} 
can be expressed in terms of our generic {\tt rank} function as:
\begin{code}
hfs2nat-->default_ulimit(Ulimit),hfs2nat_(Ulimit).

hfs2nat_(Ulimit)-->rank_(Ulimit,set2nat).
\end{code}
where the constant provided by {\tt default\_ulimit} controls the
segment {\tt [0..Ulimit-1]} of $Nat$ to be mapped to urelements. 
For each natural number $u$ this provides a generalization of Ackermann's
mapping, to hereditarily finite sets with urelements in $[0..u-1]$ defined as:
\vskip 0.30cm
$f_{u}(x)$ = 
{\tt if} $x<u$ 
{\tt then} $x$
{\tt else} $u+\sum_{a\in x}2^{f_{u}(a)}$ 
\vskip 0.30cm
For $u=0$ this becomes Ackermann's original mapping from ``pure" hereditarily
finite sets, all built from the empty set only, to natural numbers.

To obtain the inverse of the Ackermann encoding, we first define the 
inverse {\tt nat2set} of the bijection {\tt set2nat}. It decomposes a natural
number $N$ into a list of exponents of 2 (seen as bit positions equaling 1 
in $N$'s bitstring representation, in increasing order).
\begin{code}
nat2set(N,Xs):-nat2elements(N,Xs,0).

nat2elements(0,[],_K).
nat2elements(N,NewEs,K1):-N>0,
  B is /\(N,1),
  N1 is N>>1,K2 is K1+1,
  add_el(B,K1,Es,NewEs),
  nat2elements(N1,Es,K2).

add_el(0,_,Es,Es).
add_el(1,K,Es,[K|Es]).
\end{code}
The inverse of the  Ackermann encoding, with urelements in {\tt [0..Ulimit-1]}
and {\tt Ulimit} mapped to {\tt []} follows:
\begin{code}
nat2hfs_(Ulimit)-->unrank_(Ulimit,nat2set).

nat2hfs-->default_ulimit(Ulimit),nat2hfs_(Ulimit).
\end{code}
We can represent the action of a hylomorphism unfolding a natural number into
a hereditarily finite set as a directed graph with outgoing edges
induced by by applying the inverse of the Ackermann encoding as shown
in Fig. \ref{f1}.
\FIG{f1}{2008 as a HFS}{0.60}{isof1.pdf}
Using an equivalent functional notation, the following proposition summarizes
the results in this subsection:
\begin{prop}
Given id = $\lambda x.x$, the following function equivalences hold:

\begin{equation}
nat2set \circ set2nat \equiv id \equiv set2nat \circ nat2set
\end{equation}

\begin{equation}
nat2hfs \circ hfs2nat \equiv id \equiv hfs2nat \circ nat2hfs
\end{equation}
\end{prop}

\section{Encoding finite permutations} \label{perm}
To obtain an encoding for finite permutations
we will first review a ranking/unranking mechanism for permutations that
involves an unconventional numeric representation, {\em factoradics}.

\subsection{The factoradic numeral system}
The factoradic numeral system \cite{knuth_art_1997-1} replaces digits
multiplied by power of a base $N$ with digits that multiply successive values
of the factorial of $N$. In the increasing order variant {\tt fr} the first
digit $d_0$ is 0, the second is $d_1 \in \{0,1\}$ and the $N$-th is $d_N \in
[0..N-1]$. The left-to-right, decreasing order variant {\tt fl} 
is obtained by reversing the digits of {\tt fr}.
\begin{codex}
?- fr(42,R),rf(R,N).
R = [0, 0, 0, 3, 1],
N = 42

?- fl(42,R),lf(R,N).
R = [1, 3, 0, 0, 0],
N = 42
\end{codex}
\noindent The Prolog predicate {\tt fr} handles the special case for $0$ and
calls {\tt fr1} which recurses and divides with increasing values of N
while collecting digits with {\tt mod}:
\begin{code}
% factoradics of N, right to left
fr(0,[0]).
fr(N,R):-N>0,fr1(1,N,R).
   
fr1(_,0,[]).
fr1(J,K,[KMJ|Rs]):-K>0,KMJ is K mod J,J1 is J+1,KDJ is K // J,
  fr1(J1,KDJ,Rs).
\end{code}
The reverse {\tt fl}, is obtained as follows:
\begin{code}
fl(N,Ds):-fr(N,Rs),reverse(Rs,Ds).
\end{code}
The predicate {\tt lf} (inverse of {\tt fl}) converts back to decimals by
summing up results while computing the factorial progressively:
\begin{code}
lf(Ls,S):-length(Ls,K),K1 is K-1,lf(K1,_,S,Ls,[]).

% from list of digits of factoradics, back to decimals
lf(0,1,0)-->[0].
lf(K,N,S)-->[D],{K>0,K1 is K-1},lf(K1,N1,S1),{N is K*N1,S is S1+D*N}.
\end{code}
Finally, {\tt rf}, the inverse of {\tt fr} is obtained by reversing {\tt fl}.
\begin{code}
rf(Ls,S):-reverse(Ls,Rs),lf(Rs,S).
\end{code}

\subsection{Ranking and unranking permutations of given size with Lehmer codes
and factoradics} 
The Lehmer code of a permutation $f$ of size $n$ is defined as the sequence
$l(f)=(l_1(f) \ldots l_i(f) \ldots l_n(f))$ 
where $l_i(f)$ is the number
of elements of the set $\{j>i|f(j)<f(i)\}$
\cite{DBLP:journals/dmtcs/MantaciR01}.
 \begin{prop}
 The Lehmer code of a permutation determines the permutation uniquely.
 \end{prop} 
The predicate {\tt perm2nth} computes a {\tt rank} 
for a permutation {\tt Ps} of {\tt Size>0}. 
It starts by first computing its Lehmer code {\tt Ls} with 
{\tt perm\_lehmer}. Then  it associates a unique natural 
number {\tt N} to {\tt Ls}, 
by converting it with the predicate {\tt lf} 
from factoradics to decimals. 
Note that the Lehmer code {\tt Ls} is used as the list of digits
in the factoradic representation.
\begin{code}
perm2nth(Ps,Size,N):-
  length(Ps,Size),
  Last is Size-1,
  ints_from(0,Last,Is),
  perm_lehmer(Is,Ps,Ls),
  lf(Ls,N).
\end{code}
The generation of the Lehmer code is surprisingly
simple and elegant in Prolog. We just instrument the
usual backtracking predicate generating a permutation
to remember the choices it makes, 
in the auxiliary predicate {\tt select\_and\_remember}!
\begin{code}
% associates Lehmer code to a permutation 
perm_lehmer([],[],[]).
perm_lehmer(Xs,[X|Zs],[K|Ks]):-
  select_and_remember(X,Xs,Ys,0,K),
  perm_lehmer(Ys,Zs,Ks).

% remembers selections - for Lehmer code
select_and_remember(X,[X|Xs],Xs,K,K).
select_and_remember(X,[Y|Xs],[Y|Ys],K1,K3):-K2 is K1+1,
  select_and_remember(X,Xs,Ys,K2,K3).
\end{code}  

The predicate {\tt nat2perm} provides the matching {\em unranking}
operation associating a permutation {\tt Ps} to a given {\tt Size>0} 
and a natural number {\tt N}.
\begin{code}
nth2perm(Size,N, Ps):-
  fl(N,Ls),length(Ls,L),
  K is Size-L,
  Last is Size-1,
  ints_from(0,Last,Is),
  zeros(K,Zs),
  append(Zs,Ls,LehmerCode),
  perm_lehmer(Is,Ps,LehmerCode).
\end{code}
Note also that {\tt perm\_lehmer} is used (reversibly!) this time
to reconstruct the permutation {\tt Ps} from its Lehmer code.
The Lehmer code is computed from the permutation's 
factoradic representation obtained by converting {\tt N} to {\tt Ls} 
and then padding it with {\tt 0}'s.
One can try out this bijective mapping as follows:
\begin{codex}
?- nth2perm(5,42,Ps),perm2nth(Ps,Length,Nth).
Ps = [1, 4, 0, 2, 3],
Length = 5,
Nth = 42

?- nth2perm(8,2008,Ps),perm2nth(Ps,Length,Nth).
Ps = [0, 3, 6, 5, 4, 7, 1, 2],
Length = 8,
Nth = 2008
\end{codex}

\subsection{A bijective mapping from permutations to $Nat$}
One more step is needed to to extend the mapping between permutations of a
given length to a bijective mapping from/to $Nat$: we will have to ``shift
towards infinity'' the starting point of each new bloc of permutations in $Nat$
as permutations of larger and larger sizes are enumerated.

First, we need to know by how much - so we compute the sum of
all factorials up to $N!$.
\begin{code}
% fast computation of the sum of all factorials up to N!
sf(0,0).
sf(N,R1):-N>0,N1 is N-1,ndup(N1,1,Ds),rf([0|Ds],R),R1 is R+1.
\end{code}
This is done by noticing that the factoradic representation of
[0,1,1,..] does just that. The stream of all such sums can now
be generated as usual:
\begin{code}
sf(S):-nat(N),sf(N,S).
\end{code}
What we are really interested into, is decomposing {\tt N} into
the distance to the
last sum of factorials smaller than {\tt N}, {\tt N\_M}
and its index in the sum, {\tt K}.
\begin{code}
to_sf(N, K,N_M):-nat(X),sf(X,S),S>N,!,K is X-1,sf(K,M),N_M is N-M.
\end{code}
{\em Unranking} of an arbitrary permutation is now easy - the index {\tt K}
determines the size of the permutation and {\tt N\_M} determines
the rank. Together they select the right permutation with {\tt nth2perm}.
\begin{code}
nat2perm(0,[]).
nat2perm(N,Ps):-to_sf(N, K,N_M),nth2perm(K,N_M,Ps).
\end{code}
{\em Ranking} of a permutation is even easier: we first compute
its {\tt Size} and its rank {\tt Nth}, then we shift the rank by 
the sum of all factorials up to {\tt Size}, enumerating the
ranks previously assigned.
\begin{code}
perm2nat([],0).
perm2nat(Ps,N) :-perm2nth(Ps, Size,Nth),sf(Size,S),N is S+Nth.
\end{code}
\begin{codex}
?- nat2perm(2008,Ps),perm2nat(Ps,N).
Ps = [1, 4, 3, 2, 0, 5, 6],
N = 2008
\end{codex}
As finite bijections are faithfully represented by permutations,
this construction provides a bijection from $Nat$ to 
the set of Finite Bijections.
\begin{prop}
The following function equivalences hold:
\begin{equation}
nat2perm \circ perm2nat \equiv id \equiv perm2nat \circ nat2perm
\end{equation}
\end{prop}

\section{Hereditarily finite permutations} \label{hfp}

By using the generic {\tt unrank\_} and {\tt rank} predicates defined 
in section \ref{unrank} we can extend the {\tt nat2perm} and {\tt perm2nat} to
encodings of hereditarily finite permutations ($HFP$).
\begin{code}
nat2hfp --> default_ulimit(D),nat2hfp_(D).   
nat2hfp_(Ulimit) --> unrank_(Ulimit,nat2perm).
hfp2nat --> rank(perm2nat).
\end{code}
The encoding works  as follows:
\begin{codex}
?- nat2hfp(42,H),hfp2nat(H,N),write(H),nl.
H = [[], [[], [[]]], [[[]], []], [[]], [[], [[]], [[], [[]]]]],
N = 42
?- nat2hfp(2008,S),write(S),nl,fail.
[[[]], [[], [[]], [[], [[]]]], [[[]], []], [[], [[]]], [], 
       [[], [[], [[]]], [[]]], [[[]], [], [[], [[]]]]].
\end{codex}
\begin{prop}
The following function equivalences hold:
\begin{equation}
{nat2hfp} \circ {hfp2nat} \equiv id \equiv {hfp2nat} \circ {nat2hfp}
\end{equation}
\end{prop}

Sets represent ``content'' in a pure way - order is immaterial. Permutations
represent ``order'' in a pure way - what is actually ordered is immaterial.
Let us note that a similar ``fractal'' structure is shared when natural
number encodings of both sets and permutations are expanded recursively
as HFSs and HFPs.

As shown in Fig \ref{f3} an ordered digraph (with labels starting from 0
representing the order of outgoing edges) can be used to represent the unfolding
of a natural number to the associated hereditarily finite permutation.
\FIG{f3}{2008 as a HFP}{0.60}{isof3.pdf}
Note that as this mapping generates
sequences where the order of the edges matters, therefore order is
indicated by labeling the edges with integers starting from {\tt 0}.
An interesting property of graphs associated to hereditarily finite permutations
is that moving from a number n to its successor typically only induces a
reordering of the labeled edges, as shown in Fig. \ref{f4}.
\FIG{f4}{2009 as a HFP}{0.60}{isof4.pdf}

It is interesting to see how ``information density'' of HFS and HFP compares.
Intuitively that would answer the question: which is more efficient -
codifying information as pure ``content'' or as pure ``order''?
 
%In a sense they are extreme representations: sets focus exclusively on
%``content'' while permutations focus exclusively on ``order''.

Figs. \ref{f9} and \ref{isoHfsHfp17} compare sizes of HFS and HFP trees obtained from
the same natural number up to $2^{10}$ and $2^{17}$ respectively. 
\FIG{f9}{Comparison of curve1=HFS and curve2=HFP sizes up to
$2^{10}$}{0.40}{isof9.pdf}

\FIG{isoHfsHfp17}
{Comparison of curve1=HFS and curve2=HFP sizes up to
$2^{17}$}{0.40}{isoHfsHfp17.pdf}

We leave the study of the relative asymptotic behavior of the two curves 
as an example of interesting {\em open problem} derived from our data type
hylomorphisms.


\begin{comment}
\section{An application: semantic steganography}
Steganography is a technique that conceals information by adding to media
objects like JPG files or MP3 music tracks a few bits encoding a secret message.
While the human eye or ear usually sees no differences in the images or sounds,
statistical techniques can be used to identify a steganographic load as a result of
as distortions in typical distributions.
Semantic variants of steganography also exist, for
instance encoding a load by replacing words with their synonyms in
a sentence or selecting or wearing a particular
dress code in a picture or performing a particular sequence of actions
in a clip. As they involve complex common sense knowledge processing steps such
techniques are unlikely to be subject to statistical detection in any close
future.

Hereditarily finite permutations can provide unusually strong
steganographic encodings without requiring 
any change in the set of objects or events encoding the load.
Such encoding can be achieved by
using solely the (recursive) ordering of naturally occurring observables
in images or texts or behaviors. And, in contrast to ``content'' order is in  a
way invisible - it is just in the eye of the beholder.

For instance the fact that on a day Alice ([]) and Bob ([[]]) visits first a
chatroom discussing about dating in the arrival order [[]],[] then leave it in order [],[[]] and
visit a chatroom about sport cars in arrival order [],[[]] and leave in order
[],[[]] and then on the next day visit first the chatroom on sport cars in order [],[[]]
leaving in order [[]],[] and then the chatroom about dating in arrival 
order [[]],[] leaving in order [[]],[] encodes the HFP:
\begin{code}
secret([[[[[]],[]],[[],[[]]]],[[[],[[]]],[[],[[]]]],[],[],[[[],[[]]],[[[]],[]]]]).
\end{code}


Assuming both are usual visitors of the two chatrooms, well known to just talk
as much nonsense as anyone else - no content or behavioral scanning of any of
the chatrooms can indicate that the hereditarily finite permutation is the
encoding of a bank account where they have deposited drug money to be
laundered by their friend Carlos.

Besides steganography, similar techniques can be used to provide invisible
digital watermarks to assert ownership of media content.
\end{comment}

\section{Related work} \label{related}
Natural number encodings of hereditarily finite sets have 
triggered the interest of researchers in fields ranging from 
axiomatic set theory and foundations of logic to 
complexity theory and combinatorics
\cite{finitemath,kaye07,abian78,DBLP:journals/jct/MeirMM83,avigad97}. 
Computational and data representation aspects of Finite Set Theory 
have been described in logic programming and theorem proving contexts 
in \cite{dovier00comparing,DBLP:journals/tplp/PiazzaP04,DBLP:conf/types/Paulson94}. 
While finite permutations have been used extensively in various branches of
mathematics and computer science, we have not seen any formalization of 
hereditarily finite permutations as such in the literature.

\section{Conclusion and Future Work} \label{concl}

We have shown the expressiveness of logic programming as a
metalanguage for executable mathematics, by describing
ranking/unranking functions
for finite sets and permutations and by extending them in a
generic way to hereditarily finite sets
and hereditarily finite permutations.

We also foresee interesting applications in cryptography and steganography. 
For instance, in the case of the permutation related encodings -  something as
simple as the order of the cities visited or the order of names 
on a greetings card, seen as a permutation with respect to their 
alphabetic order, can provide a steganographic encoding/decoding of a secret
message by using predicates like {\tt nat2perm} and {\tt perm2nat}.

Last but not least, the use of a logic programming language to express
in a generic way some fairly intricate combinatorial algorithms
predicts an interesting new application area.

\bibliographystyle{plain}
%\bibliographystyle{plainnat}
\bibliography{INCLUDES/theory,tarau,INCLUDES/proglang,INCLUDES/biblio,INCLUDES/syn}

\newpage
\appendix
\section{Appendix}
To make the code in the paper fully self contained, 
we list here some auxiliary predicates.

\begin{code}
% generates integers From..To
ints_from(From,To,Is):-findall(I,between(From,To,I),Is).

% replicates X, N times
ndup(0, _,[]).
ndup(N,X,[X|Xs]):-N>0,N1 is N-1,ndup(N1,X,Xs).
  
zeros(N,Zs):-ndup(N,0,Zs).

% generator for the stream of natural numbers 0,1,2,...
nat(0).
nat(N):-nat(N1),N is N1+1.  
\end{code}

The following predicates print out a $HFS$ or $HFP$ with {\em Urelements}
\begin{code}
show_hfs(S):-gshow(S,"{,}"),nl.
show_hfp(S):-gshow(S,"( )"),nl.

gshow(0,[L,_C,R]):-put(L),put(R).
gshow(N,_):-integer(N),N>0,!,write(N).
gshow(Hs,[L,C,R]):-put(L),gshow_all(Hs,[L,C,R]),put(R).

gshow_all([],_).
gshow_all([H],LCR):-gshow(H,LCR).
gshow_all([H,G|Hs],[L,C,R]):-
  gshow(H,[L,C,R]),
  ([C]\=="~"->put(C);true),
  gshow_all([G|Hs],[L,C,R]).
\end{code}
as shown in the following examples: 
\begin{codex}
?- nat2hfs(2009,H),show_hfs(H).
{{},{{},{{}}},{{{{}}}},{{{}},{{{}}}},{{},{{}},{{{}}}},
    {{{},{{}}}},{{},{{},{{}}}},{{{}},{{},{{}}}}}

?- nat2hfp(2009,H),show_hfp(H).
((()) (() (()) (() (()))) ((()) ()) (() (())) () 
 ((()) () (() (()))) (() (() (())) (())))
 
?- assert(ulimit(4)).
true.

?- nat2hfs(2009,H),show_hfs(H).
{{},2,{},{1},{{},1},{2},{{},2},{1,2}}

?- nat2hfp(2009,H),show_hfp(H).
(1 () 3 () (()) (() 1) 2)
 \end{codex}
\end{document}
