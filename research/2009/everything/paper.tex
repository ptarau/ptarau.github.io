\pdfoutput=1
% ICFP'09 Confirmation Number:    104
% Submission Passcode:    104X-H1C5C0J9I2 

\documentclass[9pt,preprint,natbib]{TOOLS/sigplanconf}
\usepackage{amsfonts,amsmath}
\input TOOLS/chheader.tex

\begin{document}
\conferenceinfo{ICFP}{date, City.} 
\copyrightyear{2009} 
\copyrightdata{[to be supplied]} 

\titlebanner{}      % These are ignored unless
\preprintfooter{}   % 'preprint' option specified.

\title{
   ``Everything'' - as a functional program
}
\subtitle{or {\em a rambling renaissance man's guide to data transformations}}
           
\authorinfo{Paul Tarau}
   {Department of Computer Science and Engineering\\
   University of North Texas}
   {\em tarau@cs.unt.edu}

\maketitle

\begin{abstract}
We explore in this paper, in the form of a literate Haskell program a uniform
data centric upper ontology of computational entities, derived from first
principles: a {\em groupoid of isomorphisms} between fundamental data types
provides the ability to shift views of a given informational entity at will.

After introducing isomorphisms between elementary
data types (natural numbers, sets, multisets, finite functions, graphs,
hypergraphs, bitstring and parenthesis languages, etc.) we lift them
to hereditarily
finite universes through {\em hylomorphisms} derived
from {\em ranking/unranking} and
{\em pairing/unpairing} operations.

An embedded higher order {\em combinator language} provides
any-to-any encodings automatically.

Applications range from
stream iterators on
combinatorial objects to
self-delimiting codes,
succinct data representations and generation
of random instances.

\end{abstract}

%\category{CR-number}{subcategory}{third-level}

%\terms


\keywords
computational mathematics in Haskell, 
data type transformations,
higher order combinators,  
hylomorphisms, 
ranking/unranking, Goedel numberings

\begin{comment}
\begin{code}
module ISO where
import Data.List
import Data.Char
import Random
-- import Data.Graph.Inductive
\end{code}
\end{comment}

\section{Introduction}

Mathematical theories often borrow proof patterns and reasoning
techniques across close and sometime not so close fields.

A relatively small number of universal data types are used as basic building
blocks in programming languages and their runtime interpreters,
corresponding to a few well tested mathematical abstractions like sets,
functions, graphs, groups, categories etc.

Compilers convert programs from human centered to machine centered
representations. Complexity classes are defined through compilation with
limited resources (time or space) to similar problems
\cite{Cook04theoriesfor}.

Clearly, this as part of a more general pattern: analogical/metaphorical
thinking routinely shifts entities and operations 
from a field to another hoping to uncover similarities
in representation or use \cite{lakoff}. 

This might sound somewhat interesting - but it is awfully vague. 
From where a precise challenge is derived: can we turn the {\em
metaphors we leave by} into well defined data transformation mechanisms?

One can see the ``computational universe'' - in analogy with its
physical or biological counterparts - as being populated by a wide diversity of
data types - all fundamentally {\em different}. Under this view, knowledge
appears as accumulation of information about similarities and differences and
 {\em abstraction} (i.e. lossy but sufficient approximation) helps managing
the otherwise intractable set of attributes and behaviors.

However, this view of the ``computational universe" forgets a
salient difference: while we have not designed the representations and laws
governing the physical or biological world, we have designed their
computational counterparts. And ownership has its privileges: we can 
freely shift the view while fixing the object of the view, if we wish!

This paper is about advocating constructively\footnote{
  We really mean it - the
  paper is executable as a literate Haskell program - the code is available
  from \url{http://logic.cse.unt.edu/tarau/research/2009/everything.zip}.
} 
an exploration of the ``computational universe'' in which everything {\em is} 
everything else, or, for the relativist, everything {\em can be seen} as everything else.

Under this assumption on the computational {\em upper ontology}, information
can take the {\em shape} of an arbitrary data type. And one is quickly conduced
to accept that the {\em frame of reference} is unimportant - and that any of these shapes 
can be seen as the essential shape of an entity, given that 
{\em isomorphisms} can faithfully shift it from one shape 
to another and back without loss of information.

We see these encodings as a first step
towards a ``theory of everything'' joining together
the basic building blocks of various computational artifacts, i.e.
an executable {\em upper ontology} of computational entities.

The cognoscenti might observe that this is not very far from what Leibniz, in
{\em La Monadologie} \cite{monadologie} had already expressed about \ldots {\em
Monads}:
\begin{quote} {\em
Now this interconnectedness, or this accommodation of all created things to
each, and of each to all the rest, means that each simple substance has relations 
to all the others, which it expresses. 
Consequently, it is a permanent living mirror of the universe.}
\end{quote}

As we shall see, the natural framework to organize such isomorphisms 
is a {\em finite groupoid} i.e. a category \cite{matcat} where every morphism is
an isomorphism, with objects provided by the data types and morphisms 
provided by their transformations.

As a practical goal, we hope this can facilitate the refactoring of the
enormous ontology exhibited by various computer science and engineering fields,
that have resulted over a relatively short period of
evolution in unnecessarily steep learning curves limiting communication
and synergy between fields. % and inducing high costs for specialized training.

It is important to keep in mind that we must not lose any
information if we want to be able to shift views at will. Therefore, the only {\em morphisms}
that matter, in this ontology, are {\em isomorphisms}. 

This brings us into
familiar territory: substitution of equals by equals - {\em referential
transparency} - an intrinsic feature of functional programming languages.
And it makes Haskell the natural choice for the first iteration of such an
effort - while keeping in mind that the frame of reference can be easily
shifted to a different programming paradigm.

\section{A possible first frame of reference: a ranking/unranking algorithm for
finite sequences} \label{cons}

One has to start somewhere - and that usually comes with a set of assumptions
that are genuinely {\em a priori} i.e. such that most people have no trouble
working with them even outside formal computer science or mathematical training.

In this sense, natural numbers are probably not just
the oldest computational abstraction - but also the most 
widely accepted one as being ``built from first principles''. With their
usual (second order) Peano axiomatization \cite{kirby82}, 
natural numbers are also formally well understood - if one cares.
We will, therefore, use them as our first data type, providing 
not only a surprisingly strong structuring mechanism - but
also enough expressiveness for hosting universal data and program
structures.
\begin{df}
A {\em ranking/unranking} function defined on a data type is a
bijection to/from the set of natural numbers (denoted $\mathbb{N}$ through the paper).
\end{df}
When applied to formulae or proofs, ranking functions are usually called
{\em G\"{o}del numberings} as they have originated in arithmetization
techniques used in the proof of G\"{o}del's incompleteness results.
G\"{o}del
numberings are typically {\em one-to-one} but not {\em onto} i.e. not all
natural numbers correspond to valid formulae. 
Originating in combinatorics, {\em ranking/unranking}
functions are {\em bijective} as they relate arbitrary combinatorial 
objects to unique natural numbers.

We start with an unusually simple (but nevertheless {\em new})
ranking/unranking algorithm for finite sequences of arbitrary (i.e. unbounded size!) 
natural numbers. Let's first observe that
\begin{prop}
$\forall z \in \mathbb{N}-\{0\}$ the diophantic equation
\begin{equation}
2^x(2y+1)=z
\end{equation}
has exactly one solution $x,y \in \mathbb{N}$.
\end{prop}
This follows immediately from the unicity of the decomposition of a natural
number as a multiset of prime factors. 

Given the definitions
\begin{code}
type N = Integer
cons :: N->N->N
cons x y  = (2^x)*(2*y+1)

hd :: N->N
hd n | n>0 = if odd n then 0 else 1+hd  (n `div` 2)

tl :: N->N
tl n = n `div` 2^((hd n)+1)

as_ns_n :: N->[N]
as_ns_n 0 = []
as_ns_n n = hd n : as_ns_n (tl n)
 
as_n_ns :: [N]->N  
as_n_ns [] = 0
as_n_ns (x:xs) = cons x (as_n_ns xs)
\end{code}
the following holds:
\begin{prop}
{\tt as\_n\_ns} is a bijection from finite sequences of natural numbers to
natural numbers and {\tt as\_ns\_n} is its inverse.
\end{prop}
This follows from the fact that {\tt cons} and the pair {\tt (hd, tl)} define a bijection between
$\mathbb{N}-\{0\}$ and $\mathbb{N} \times \mathbb{N}$ and that the value of {\tt as\_n\_ns} is
uniquely determined by the applications of {\tt tl} and the sequence of values
returned by {\tt hd}.
\begin{codex}
*ISO> hd 2008
3
*ISO> tl 2008
125
*ISO> cons 3 125
2008
\end{codex}
Note also that this isomorphism preserves ``list processing'' operations i.e.
if one defines:
\begin{code}
append 0 ys = ys
append xs ys = cons (hd xs) (append (tl xs) ys)
\end{code}
then the isomorphism commutes with operations like list concatenation:
\begin{prop}
~\\
(as\_ns\_n n)++(as\_ns\_n m) $\equiv$ as\_ns\_n (append n m)\\
as\_n\_ns (ns++ms) $\equiv$ append (as\_n\_ns ns) (as\_n\_ns ms)
\end{prop}

One might notice at this point that {\tt hd,tl,cons,0}
define on $\mathbb{N}$ an algebraic structure isomorphic to the one 
introduced by {\tt CAR,CDR,CONS,NIL} in John McCarthy's classic LISP paper
\cite{mccarthy60}\footnote{And following John McCarthy's {\em eval} construct
one can now build relatively easily a LISP interpreter working directly and
exclusively through simple arithmetic operations on natural numbers.}.

\section{Connecting the dots with a groupoid of isomorphisms}

\begin{df}
A category in which every morphism is an isomorphism is called a {\em groupoid}.
\end{df}
We represent {\em isomorphism} pairs like {\tt as\_ns\_n} and {\tt
as\_n\_ns} as a data type {\tt Iso}, together with the operations  {\tt compose,
itself, invert} providing the (finite) {\em groupoid} structure.�

\begin{code}
data Iso a b = Iso (a->b) (b->a)

compose :: Iso a b -> Iso b c -> Iso a c
compose (Iso f g) (Iso f' g') = Iso (f' . f) (g . g')

itself = Iso id id

invert (Iso f g) = Iso g f
\end{code}
It makes sense at this point to connect everything to a {\em hub} type --
for instance $\mathbb{N}$ -- to avoid having to provide $n*(n-1)/2$
isomorphisms. We call such a connector an {\tt Encoder}:
\begin{code}
type Encoder a = Iso a N
\end{code}
We first define a trivial {\tt Encoder}:
\begin{code}
n :: Encoder N
n = itself
\end{code}
and then an {\tt Encoder} from finite sequences of (unbounded) natural numbers
to $\mathbb{N}$:
\begin{code}
ns :: Encoder [N] 
ns = Iso as_n_ns as_ns_n
\end{code}
It makes sense to lift the operations  {\tt as\_ns\_n} and {\tt
as\_n\_ns} to route a transformation to/from an arbitrary type through the
common {\em hub} by introducing a new combinator {\tt as}:
\begin{code}  
as :: Encoder a -> Encoder b -> b -> a
as that this x = g x where 
   Iso _ g = compose that (invert this)
\end{code}
It works as follows:
\begin{codex}
*ISO> as ns n 2009
[0,2,0,1,0,0,0,0]
*ISO> as n ns [0,2,0,1,0,0,0,0]
2009
\end{codex}
Note that indeed, {\tt as ns n} generalizes {\tt as\_ns\_n} in an intuitive way:
\begin{codex}
*ISO> as_ns_n 2009
[0,2,0,1,0,0,0,0]
*ISO> as_n_ns [0,2,0,1,0,0,0,0]
2009
\end{codex}

\section{Encoding of some fundamental data types}
We will now quickly put the mechanism at work and show that Encoders for some
fundamental data types are surprisingly easy to build.

\subsection{From finite sequences to finite sets of natural numbers}
An encoder of finite sets of natural numbers (assumed ordered) as sequences is
obtained by adjusting the encoding of multisets so that {\tt 0}s are first mapped to {\tt 1}s - this
ensures that all elements are different.
\begin{code}
set :: Encoder [N]
set = compose (Iso as_ns_set as_set_ns) ns

as_set_ns = (map pred) . as_ms_ns . (map succ)
as_ns_set = (map pred) . as_ns_ms . (map succ)
\end{code}
It works as follows:
\begin{codex}
*ISO> as set n 2009
[0,3,4,6,7,8,9,10]
*ISO> as n set it
2009
\end{codex}

\subsection{From sequences to multisets of natural numbers}
An encoder of multisets (assumed ordered) as sequences is obtained by ``summing
up'' a sequence with {\tt scanl}.
\begin{code}
ms :: Encoder [N]
ms = compose (Iso as_ns_ms as_ms_ns) ns

as_ms_ns ns = tail (scanl (+) 0 ns)
as_ns_ms ms = zipWith (-) (ms) (0:ms)
\end{code}
It works as follows:
\begin{codex}
*ISO> as ms n 2009
[0,2,2,3,3,3,3,3]
*ISO> as n ms it
2009
\end{codex}
One can see how this is derived by summing up from:
\begin{codex}
*ISO> as ns n 2009
[0,2,0,1,0,0,0,0]
\end{codex}



\section{Generic unranking and ranking hylomorphisms} \label{unrank}

The {\em ranking problem} for a family of
combinatorial objects is finding a unique 
natural number associated to it,
called its {\em rank}.
The inverse {\em unranking problem} consists of generating a unique
combinatorial object associated to each natural number. 

\subsection{Hereditarily finite data types}

The unranking operation is seen here as an instance of a generic
{\em anamorphism} (an {\em unfold} operation), while the ranking
operation is seen as an instance of the corresponding catamorphism (a {\em
fold} operation) \cite{DBLP:conf/fpca/MeijerH95,DBLP:journals/jfp/Hutton99}.
Together they form a mixed transformation called {\em hylomorphism}. 
We will use such hylomorphisms to {\em lift} isomorphisms between lists
and natural numbers to isomorphisms between a derived tree
data type and natural numbers.

The data type representing such {\em hereditarily finite} structures will be
a generic multi-way tree with a single leaf type {\tt []}.

\begin{code}
data T = H Ts deriving (Eq,Ord,Read,Show)
type Ts = [T]
\end{code}
The two sides of our hylomorphism 
are parameterized by two transformations
{\tt f} and {\tt g} forming
an isomorphism {\tt Iso f g}:
\begin{code}
unrank f n = H (unranks f (f n))
unranks f ns = map (unrank f) ns

rank g (H ts) = g (ranks g ts)
ranks g ts = map (rank g) ts
\end{code}
Both combinators can be seen as a form of ``structured recursion''
that propagates a simpler operation guided by the structure of
the data type. 
We can now combine an 
anamorphism+catamorphism pair into
an isomorphism {\tt hylo} defined
with {\tt rank} and {\tt unrank} 
on the corresponding
hereditarily finite data types:
\begin{code}
hylo :: Iso b [b] -> Iso T b
hylo (Iso f g) = Iso (rank g) (unrank f)

hylos :: Iso b [b] -> Iso Ts [b]
hylos (Iso f g) = Iso (ranks g) (unranks f)
\end{code}

\subsubsection{Hereditarily finite sets}
Hereditarily finite sets \cite{finitemath} will be represented
as an Encoder for the tree type {\tt T}:
\begin{code}
hfs :: Encoder T
hfs = hylo (Iso (as set n) (as n set))
\end{code}
Otherwise, hylomorphism induced isomorphisms
work as usual with our embedded
transformation language:

\begin{codex}
*ISO> as hfs n 42
H [H [H []],H [H [],H [H []]],H [H [],H [H [H []]]]]
\end{codex}
One can notice that we have just derived as a
``free algorithm'' Ackermann's encoding
\cite{ackencoding}
from hereditarily finite sets to natural numbers:
\vskip 0.30cm
$f(x)$ = {\tt if} $x=\{\}$ {\tt then} $0$ {\tt else} $\sum_{a \in x}2^{f(a)}$
\vskip 0.30cm
\noindent together with its inverse
\begin{code}
ackermann = as n hfs
inverse_ackermann = as hfs n
\end{code}
One can represent the action of a hylomorphism unfolding a natural number into
a hereditarily finite set as a directed graph with outgoing edges
induced by by applying the {\tt inverse\_ackermann} function as shown
in Fig. \ref{f1}.
\FIG{f1}{2008 as a HFS}{0.50}{isof1.pdf}

\subsection{Hereditarily finite multisets}
In a similar way, one can derive an Encoder for hereditarily finite multisets
based on the {\tt ms} isomorphism:
\begin{code}
hfm :: Encoder T
hfm = hylo (Iso (as ms n) (as n ms))
\end{code}
working as follows:
\begin{codex}
*ISO> as hfm n 2008
H [H [H [],H []],H [H [],H []],
   H [H [H [H []]]],H [H [H [H []]]],
   H [H [H [H []]]],H [H [H [H []]]],H [H [H [H []]]]]
*ISO> as n hfm it
2008
\end{codex}

\subsubsection{Hereditarily finite functions}
The same tree data type can host a hylomorphism
derived from finite functions instead of
finite sets:
\begin{code}
hff :: Encoder T
hff = hylo (Iso (as ns n) (as n ns))
\end{code}
The {\tt hff} Encoder can be seen as a ``free algorithm'', providing
data compression/succinct representation for Hereditarily 
Finite Sets. Note, for instance,
the significantly smaller tree size in:
\begin{codex}
*ISO> as hff n 2008
H [H [H [],H []],H [],H [H []],H [],H [],H [],H []]
\end{codex}
One can represent the action of a hylomorphism unfolding a natural number into
a hereditarily finite function as a directed ordered multi-graph as shown
in Fig. \ref{f2}. Note that as the mapping {\tt as fun n} generates
a sequence where the order of the edges matters, this order is
indicated integers starting from {\tt 0} labeling the edges.
\FIG{f2}{2008 as a HFF}{0.50}{isof2.pdf}

\section{Pairing functions as Encoders}
An important type of isomorphism, originating in Cantor's work of infinite sets
connects natural numbers and pairs of natural numbers.
\begin{df}
An isomorphism $f: \mathbb{N} \times \mathbb{N} \rightarrow \mathbb{N}$ is
called a {\em pairing function} and its inverse $f^{-1}$ is called an {\em
unpairing} function.
\end{df}
Given the definitions:
\begin{code}
unpair z = (hd (z+1), tl (z+1))
pair (x,y) = (cons x y)-1
\end{code}
shifting by 1 turns {\tt hd} and {\tt tl} in
total functions on $\mathbb{N}$ such that $unpair~0=(0,0)$ i.e. the following
holds:
\begin{prop}
~\\
$unpair:\mathbb{N} \rightarrow \mathbb{N} \times \mathbb{N}$ 
is a bijection and $pair=unpair^{-1}$.
\end{prop}
Note that unlike {\tt hd} and {\tt tl}, {\tt
unpair} is defined for all natural numbers:
\begin{codex}
*ISO> map unpair [0..7]
[(0,0),(1,0),(0,1),(2,0),(0,2),(1,1),(0,3),(3,0)]
\end{codex}

As the cognoscenti might notice, this is in fact a classic {\em
pairing/unpairing function} that has been used, by Pepis, Kalmar and Robinson 
in some fundamental work on recursion theory, 
decidability and Hilbert's Tenth Problem in
\cite{pepis,kalmar1,robinson50,robinsons68b}.

Using the functions {\tt pair} and {\tt unpair} we define the Encoder:
\begin{code}
type N2 = (N,N)
n2 :: Encoder N2
n2 = Iso pair unpair
\end{code}
we obtain a pairing/unpairing isomorphism {\tt n2} between and $\mathbb{N}
\times \mathbb{N}$ and $\mathbb{N}$.
Another interesting pairing function \cite{pigeon} can be derived from
the Moser-De Bruijn sequence (A000695, \cite{A000695})
\begin{code}
inflate = (as n set) . (map (*2)) . (as set n)
\end{code}
as follows:
\begin{code}
bpair (i,j) = inflate i + ((*2) . inflate) j
bunpair k = (deflate xs,deflate ys) where 
  (xs,ys) = partition even (as set n k)
  deflate = (as n set) . (map (`div` 2))
\end{code}
It works as follows:
\begin{codex}
*ISO> map bunpair [0..7]
[(0,0),(1,0),(0,1),(1,1),(2,0),(3,0),(2,1),(3,1)]
*ISO> map bpair it
[0,1,2,3,4,5,6,7]
\end{codex}
\begin{prop}
{\tt bunpair} creates a pair by separating even and odd bits of a natural
number; its inverse {\tt bpair} merges them back.
\end{prop}
Together, they provide a pairing/unpairing isomorphism {\tt n2'} between and
$\mathbb{N} \times \mathbb{N}$ and $\mathbb{N}$:
\begin{code}
n2' :: Encoder N2
n2' = Iso bpair bunpair
\end{code}
working as follows
\begin{codex}
*ISO> as n n2' (2008,2009)
4191170
*ISO> as n2' n it
(2008,2009)
\end{codex}

In a way similar to hereditarily finite trees generated by unfoldings one can
apply strictly decreasing\footnote{except for 0 and 1, typically} unpairing
functions recursively. Figures \ref{iso2008p} and \ref{iso2008b} show the
directed graphs describing recursive application of {\tt bunpair} and {\tt
pepis\_unpair}.

\FIG{iso2008p}{Graph obtained by recursive application of {\tt unpair}
for 2008}{0.40}{iso2008p.pdf}

\FIG{iso2008b}{Graph obtained by recursive application of {\tt bunpair}
for 2008}{0.40}{iso2008b.pdf}

Given that unpairing functions are bijections from $N$ to $N \times N$
they will progressively cover all points having natural number coordinates in
their range in the plane. Figure \ref{isounpair1} %, \ref{isounpair2} 
shows the curve generated by {\tt bunpair}. %and {\tt unpair}.

\FIG{isounpair1}
{2D curve connecting values of {\tt unpair n} for $n \in [0..2^{10}-1]$}
{0.25}{isounpair1.pdf}

\begin{comment}

\FIG{isounpair2}
{2D curve connecting values of {\tt unpair n} for $n \in [0..2^{10}-1]$}
{0.25}{isounpair2.pdf}

Fig. \ref{f6} shows the action of the pairing function {\tt bpair}
on its two arguments arguments in [0..63].
\FIG{f6}{Values of bpair x y with x,y in [0..63]}{0.25}{isof6.pdf}

\end{comment}

\subsection{Encoding unordered pairs}
To derive an encoding of unordered pairs, i.e. 2 element sets, one
can combine pairing/unpairing with conversion between sequences and
sets:
\begin{code}
pair2unord_pair (x,y) = as set ns [x,y]
unord_pair2pair [a,b] = (x,y) where 
  [x,y]=as ns set [a,b]   

unord_unpair = pair2unord_pair . bunpair
unord_pair = bpair . unord_pair2pair
\end{code}
We can derive the following  Encoder:
\begin{code}
set2 :: Encoder [N]
set2 = compose (Iso unord_pair unord_unpair) n
\end{code}
working as follows:
\begin{codex}
*ISO> as set2' n 2008
[60,87]
*ISO> as n set2' [60,87]
2008
*ISO> as n set2' [87,60]
2008
\end{codex}

\subsection{Encodings multiset pairs}
To derive an encoding of 2 element multisets, one
can combine pairing/unpairing with conversion between sequences and
multisets:
\begin{code}
pair2mset_pair (x,y) = (a,b) where [a,b]=as ms ns [x,y]
mset_unpair2pair (a,b) = (x,y) where [x,y] = as ns ms [a,b]

mset_unpair = pair2mset_pair . bunpair
mset_pair = bpair . mset_unpair2pair
\end{code}
We can derive the following Encoder:

\begin{code}
mset2 :: Encoder N2
mset2 = compose (Iso mset_unpair2pair pair2mset_pair) n2
\end{code}
working as follows:
\begin{codex}
*ISO> as mset2 n 2008
(60,86)
*ISO> as n mset2 it
2008
\end{codex}

Figure \ref{isounpair3} shows the curve generated by {\tt mset\_unpair}
covering the lattice of points in its range.

\FIG{isounpair3}
{2D curve connecting values of {\tt mset\_unpair n} for $n \in [0..2^{10}-1]$}
{0.25}{isounpair3.pdf}


\subsection{Encoding signed integers}
To encode signed integers one can map positive numbers to even numbers and
strictly negative numbers to odd numbers. This gives the Encoder:

\begin{code}
z:: Encoder Z
z = compose (Iso z2nat nat2z) n

nat2z n = if even n then n `div` 2 else (-n-1) `div` 2
z2nat n = if n<0 then -2*n-1 else 2*n
\end{code}
working as follows:
\begin{codex}
*ISO> as set z (-42)
[0,1,4,6]
*ISO> as z set [0,1,4,6]
-42
\end{codex}

\subsection{Extending pairing/unpairing to signed integers}
Given the bijection from $n$ to $z$ one can easily extend pairing/unpairing
operations to signed integers. We obtain the Encoder:
\begin{code}
type Z = Integer
type Z2 = (Z,Z)

z2 :: Encoder Z2
z2 = compose (Iso zpair zunpair) n

zpair (x,y) = (nat2z . bpair) (z2nat x,z2nat y)
zunpair z = (nat2z n,nat2z m) where (n,m)= (bunpair . z2nat) z
\end{code}
working as follows:
\begin{codex}
*ISO> map zunpair [-5..5]
[(-1,1),(-2,-1),(-2,0),(-1,-1),(-1,0),
 (0,0),(0,-1),(1,0),(1,-1),(0,1),(0,-2)]
*ISO> map zpair it
[-5,-4,-3,-2,-1,0,1,2,3,4,5]

*ISO> as z2 z (-2008)
(63,-26)
*ISO> as z z2 it
-2008
\end{codex}
Figure \ref{isozunpair} shows the curve covering the lattice of integer
coordinates generated by the function {\tt zunpair}.

\FIG{isozunpair}
{Curve generated by unpairing function on signed integers}
{0.25}{isozunpair.pdf}

The same construction can be extended to multiset pairing functions:
\begin{code}
mz2 :: Encoder Z2
mz2 = compose (Iso mzpair mzunpair) n

mzpair (x,y) = (nat2z . mset_pair) (z2nat x,z2nat y)
mzunpair z = (nat2z n,nat2z m) where 
  (n,m)= (mset_unpair . z2nat) z
\end{code}
working as follows:
\begin{codex}
*ISO> as mz2 z (-42)
(1,-8)
*ISO> as z mz2 it
-42
\end{codex}



\section{Encoding graphs and hypergraphs} \label{graphs}
We will now show that more complex data types like digraphs and hypergraphs
have extremely simple encoders. This shows once more the importance
of compositionality in the design of our embedded transformation
language.
 
\subsection{Encoding directed graphs} \label{digraphs}
We can find a bijection from 
directed graphs (with no isolated vertices, corresponding to their
 view as binary relations), to finite sets by fusing their list
 of ordered pair representation into finite sets with a pairing
 function:
\begin{code}
digraph2set ps = map bpair ps
set2digraph ns = map bunpair ns
\end{code}
The resulting Encoder is:
\begin{code}
digraph :: Encoder [N2]
digraph = compose (Iso digraph2set set2digraph) set
\end{code}
It works as follows:
\begin{codex}
*ISO> as digraph n 2008
[(1,1),(2,0),(2,1),(3,1),(0,2),(1,2),(0,3)]
*ISO> as n digraph it
2008
\end{codex}
Fig. \ref{f5} shows the digraph associated to {\tt 2008}.
\FIG{f5}{2008 as a digraph}{0.50}{isof5.pdf}


\subsection{Encoding undirected graphs} \label{ugraphs}
We can find a bijection from 
undirected graphs to finite sets by fusing their list
 of unordered pair representation into finite sets with a pairing
 function on unordered pairs:
\begin{code}
graph2set ps = map unord_pair ps
set2graph ns = map unord_unpair ns
\end{code}
The resulting Encoder is:
\begin{code}
graph :: Encoder [[N]]
graph = compose (Iso graph2set set2graph) set
\end{code}
working as follows:
\begin{codex}
*ISO> as graph n 2008
[[1,3],[2,3],[2,4],[3,5],[0,3],[1,4],[0,4]]
*ISO> as n graph it
2008
*ISO> as n graph 
        [[1,3],[3,2],[2,4],[5,3],[0,3],[4,1],[0,4]]
2008
\end{codex}
Note that, as expected, the result is invariant to changing the order of
elements in pairs like {\tt [1,4]} and {\tt [3,5]} to {\tt [4,1]} and
{\tt [5,3]}.

\subsection{Encoding directed multigraphs} \label{mdigraphs}
We can find a bijection from 
directed multigraphs (directed graphs with multiple edges between pairs of
vertices) to finite sequences by fusing their list of ordered pair
representation into finite sequences with a pairing function.
The resulting Encoder is:
\begin{code}
mdigraph :: Encoder [N2]
mdigraph = compose (Iso digraph2set set2digraph) ns
\end{code}

\begin{codex}
*ISO> as mdigraph n 2008
[(1,1),(0,0),(1,0),(0,0),(0,0),(0,0),(0,0)]
*ISO> as n mdigraph it
2008
\end{codex}
Note that the only change to the {\tt digraph} Encoder is replacing
the composition with {\tt set} by a composition with {\tt ns}.
 
\subsection{Encoding undirected multigraphs} \label{mgraphs}
We can find a bijection from 
undirected multigraphs (undirected graphs with multiple edges between 
unordered pairs of
vertices) to finite sequences by fusing their list of pair
representation into finite sequences with a 
pairing function on unordered pairs:

The resulting Encoder is:
\begin{code}
mgraph :: Encoder [[N]]
mgraph = compose (Iso graph2set set2graph) ns
\end{code}
working as follows:
\begin{codex}
*ISO> as mgraph n 2008
[[1,3],[0,1],[1,2],[0,1],[0,1],[0,1],[0,1]]
*ISO> as n mgraph it
2008
\end{codex}
Note that the only change to the {\tt graph} Encoder is replacing
the composition with {\tt set} by a composition with {\tt fun}.

\subsection{Encoding hypergraphs}
A hypergraph (also called {\em set system}) is a pair $H=(X,E)$ where
$X$ is a set and $E$ is a set of non-empty subsets of $X$.

We can easily
derive a bijective encoding of {\em hypergraphs}, 
represented as sets of sets:
\begin{code}
set2hypergraph = map (as set n)
hypergraph2set = map (as n set)
\end{code}
The resulting Encoder is:
\begin{code}
hypergraph :: Encoder [[N]]
hypergraph = compose (Iso hypergraph2set set2hypergraph) set
\end{code}
It works as follows
\begin{codex}
*ISO> as hypergraph n 2008
[[0,1],[2],[1,2],[0,1,2],[3],[0,3],[1,3]]
*ISO> as n hypergraph it
2008
\end{codex}


\section{Encoding parenthesis languages} \label{paren}
An encoder for a parenthesis language is obtained by
combining a parser and writer. As hereditarily finite
functions naturally map one-to-one to parenthesis expressions expressed as
bitstrings, we will choose them as target of the transformers.
\begin{code}
hff_pars :: Encoder [N]
hff_pars = compose (Iso pars2hff hff2pars) hff
\end{code}

The parser recurses over a bitstring and builds a {\tt HFF} as follows:
\begin{code}
pars2hff cs = parse_pars 0 1 cs

parse_pars l r cs | newcs == [] = t where
  (t,newcs)=pars_expr l r cs
  pars_expr l r (c:cs) | c==l = ((H ts),newcs) where
    (ts,newcs) = pars_list l r cs    
  pars_list l r (c:cs) | c==r = ([],cs)
  pars_list l r (c:cs) = ((t:ts),cs2) where 
    (t,cs1)=pars_expr l r (c:cs)
    (ts,cs2)=pars_list l r cs1
\end{code}
The writer recurses over a {\tt HFF} and collects
matching ``parenthesis'' (denoted 0 and 1) pairs:
\begin{code}
hff2pars = collect_pars 0 1

collect_pars l r (H ns) =
  [l]++ (concatMap (collect_pars l r) ns)++[r] 
\end{code}
This transformation maps {\tt 42} into the bitstring representation of {\tt
"((())(())(()))"} as {\tt [0,0,0,1,1,0,0,1,1,0,0,1,1,1]}.

\section{Parenthesis encoding of hereditarily finite types as a
self-delimiting code}

Like the Elias omega code \cite{elias75}, a balanced parenthesis
representation is obviously {\em self-delimiting} as proven by the fact that 
the reader {\tt pars\_expr} defined in section \ref{paren} will
extract a balanced parenthesis expression from a finite or infinite list
while returning the part of the list left over.
More precisely, the following holds:
\begin{prop}
The {\tt hff\_pars} encoding is a self-delimiting code. If $n$ is a natural
number then {\tt hd n} equals the code of the first parenthesized subexpression
of the code of $n$ and {\tt tl n} equals the code of the expression obtained by
removing it from the code for $n$, both of which represent self-delimiting codes.
\end{prop}

One can compute, for comparison purposes, the optimal undelimited
bitstring encoding provided by the Encoder {\tt bits}
\begin{codex}
*ISO2> as bits n 2008
[1,0,0,1,1,0,1,1,1,1]
*ISO> as hff_pars n 2008
[0,0,0,1,0,1,1,0,1,0,0,1,1,0,1,0,1,0,1,0,1,1]
\end{codex}
As the last example shows, the information density of
a parenthesis representation is lower than the information theoretical optimal
representation provided by the (undelimited) {\tt bits} Encoder that maps
$\mathbb{N}$ to the regular language $\{0,1\}^*$

There are however cases when the parenthesis language representation is more
compact. For instance,
\begin{codex}
*ISO> as n bits (as hff_pars n (2^2^16))
32639
\end{codex}
while the conventional representation of the same number would
have thousands of digits. We refer to \cite{arxiv:fISO} for
a comparison between this self-delimiting code and Elias omega code.

\section{Other encodings}
We will sample a few miscellaneous encodings showing the
flexibility and expressiveness of our framework.

\subsection{Encoding SAT problems}
Boolean Satisfiability (SAT) problems are encoded as lists of lists representing
conjunctions of disjunctions of positive or negative propositional symbols.

After defining:
\begin{code}
set2sat = map (set2disj . (as set n)) where
  shift0 z = if (z<0) then z else z+1
  set2disj = map (shift0. nat2z)
  
sat2set = map ((as n set) . disj2set) where
  shiftback0 z = if(z<0) then z else z-1
  disj2set = map (z2nat . shiftback0)
\end{code}
we obtain the Encoder
\begin{code}
sat :: Encoder [[Z]]
sat = compose (Iso sat2set set2sat) set
\end{code}
working as follows:
\begin{codex}
*ISO> as sat n 2008
[[1,-1],[2],[-1,2],[1,-1,2],[-2],[1,-2],[-1,-2]]
*ISO> as n sat it
2008
\end{codex}
Clearly this encoding can be used to generate random SAT problems out of
easier to generate random natural numbers.

\subsection{Encoding strings}
Strings can be seen just as a notational equivalent
of lists of natural numbers written in base $2^n$. Assuming n=128, for
ASCII strings we obtain an Encoder immediately as: 
\begin{code}
string :: Encoder String
string = Iso string2n n2string

string2n cs = from_base 128 (map (fromIntegral . ord) cs)

n2string xs = map (chr . fromIntegral) (to_base 128 xs)
\end{code}

\begin{codex}
*ISO> as ns string "hello"
[3,1,0,0,1,2,0,2,0,1,0,2,0,1,0,0,0,0,0,1,0]
*ISO> as string ns it
"hello"
\end{codex}

\subsection{Functional binary numbers}

Church numerals are well known as a functional
representation for Peano arithmetic. While
benefiting from lazy evaluation, they
implement a form of unary arithmetic
that uses $O(n)$ space to represent $n$.
This suggest devising a functional representation
that mimics binary numbers. We will do this 
following the model described in subsection \ref{bits}
to provide an isomorphism between $\mathbb{N}$
and the functional equivalent of
the regular language $\{0,1\}^*$. We will
view each bit as a $\mathbb{N} \rightarrow \mathbb{N}$ 
transformer:
\begin{code}
b x = pred x -- begin
o x = 2*x+0  -- bit 0
i x = 2*x+1  -- bit 1
e = 1        -- end
\end{code}
As the following example shows, composition
of functions $o$ and $i$
closely parallels the corresponding bitlists: 
\begin{codex}
*ISO> b$i$o$o$i$i$o$i$i$i$i$e
2008
*ISO> as bits n 2008
[1,0,0,1,1,0,1,1,1,1]
\end{codex}
%$

We can follow the same model with an abstract data type:
\begin{code}
data D = E | O D | I D deriving (Eq,Ord,Show,Read)
data B = B D deriving (Eq,Ord,Show,Read)
\end{code}
from which we can generate functional
bitstrings as an instance of a {\em fold} operation:
\begin{code}
funbits2n :: B -> N
funbits2n = bfold b o i e

bfold fb fo fi fe (B d) = fb (dfold d) where
  dfold E = fe
  dfold (O x) = fo (dfold x)
  dfold (I x) = fi (dfold x)
\end{code}
Dually, we can reverse the effect of the functions $b,o,i,e$ as:
\begin{code}
b' x = succ x
o' x | even x = x `div` 2
i' x | odd x = (x-1) `div` 2
e' = 1
\end{code}
and define a generator for our data type as an {\em unfold} operation:
\begin{code}
n2funbits :: N -> B
n2funbits = bunfold b' o' i' e'

bunfold fb fo fi fe x = B (dunfold (fb x)) where
  dunfold n | n==fe = E
  dunfold n | even n = O (dunfold (fo n))
  dunfold n | odd n = I (dunfold (fi n))
\end{code}
The two operations form an isomorphism:
\begin{codex}
*ISO> funbits2n 
   (B $ I $ O $ O $ I $ I $ O $ I $ I $ I $ I $ E)
2008
*ISO> n2funbits it
B (I (O (O (I (I (O (I (I (I (I E))))))))))
\end{codex}
%$
We can define our Encoder as follows:
\begin{code}
funbits :: Encoder B
funbits = compose (Iso funbits2n n2funbits) n
\end{code}

Arithmetic operations can now be performed directly on this representation.
For instance, one can define a successor function as:
\begin{code}
bsucc (B d) = B (dsucc d) where
  dsucc E = O E
  dsucc (O x) = I x
  dsucc (I x) = O (dsucc x) 
\end{code}
Equivalently, arithmetics can be borrowed from $\mathbb{N}$:
\begin{codex}
*ISO> bsucc (B $ I $ O $ O $ I $ I $ 
             O $ I $ I $ I $ I $ E)
B (O (I (O (I (I (O (I (I (I (I E))))))))))
*ISO> as n funbits it
2009
*ISO> borrow (with n funbits) succ (B $ I $ O $ O $ I $ 
                             I $ O $ I $ I $ I $ I $ E)
B (O (I (O (I (I (O (I (I (I (I E))))))))))
*ISO> as n funbits it
2009
\end{codex}
While Haskell's C-based arbitrary length integers are likely
to be more efficient for most operations, this representation, like Church
numerals, has the benefit of supporting partial or delayed computations
through lazy evaluation.

\section{An alternative starting point: the regular language
$\{0,1\}^{*}$} \label{bits}
As the reader might guess - this can go on and on - and in fact 
one can find in \cite{arxiv:fISO} a much larger set of encodings 
covering more than 60 data types as diverse as BDDs, 
dyadic rationals, Gauss integers, DNA strands, etc.

But an important point we want to make in this paper is that the frame of
reference is unimportant. In fact, the frame of reference used in
\cite{arxiv:fISO} is finite functions while in this paper we picked
$\mathbb{N}$ as our {\em hub}. We will now rebuild $\mathbb{N}$ itself
through a more computer oriented - and ultimately a more {\em information
theoretical} view: as arbitrary bitstrings i.e. as elements of the regular
language $\{0,1\}^{*}$. First we need a decoder/encoder:
\begin{code}
bits :: Encoder [N]
bits = Iso as_n_bits as_bits_n

as_bits_n = drop_last . (to_base 2) . succ where
  drop_last = reverse . tail . reverse

as_n_bits bs = pred (from_base 2 (bs ++ [1]))
    
to_base base n = 
  d : (if q==0 then [] else (to_base base q)) where
     (q,d) = quotRem n base
    
from_base base [] = 0
from_base base (x:xs) | x>=0 && x<base = 
   x+base*(from_base base xs)
\end{code}
It works as follows:
\begin{codex}
*ISO> as bits n 42
[1,1,0,1,0]
*ISO> as n bits [1,1,0,1,0]
42
\end{codex}
Note that the bit order is from smaller to larger exponents of 2 and that final 1
digits (used as delimiters in conventional computer representations of binary
numbers) have been removed. This ensures that every combination of 0 and 1 in
$\{0,1\}^{*}$ represents a number.

We start with successor/predecessor operations:
\begin{code}
s [] = [0]
s (0:xs) = 1:xs
s (1:xs) = 0:s xs

p [0] = []
p (0:xs) = 1:p xs 
p (1:xs) = 0:xs
\end{code}
working as expected
\begin{codex}
*ISO> as n bits (s (as bits n 42))
43
*ISO> as n bits (p (as bits n 43))
42
\end{codex}
A new design pattern emerges:
\begin{code}
nf f = (as n bits) . f . (as bits n)
nf2 f x y = as n bits (f (as bits n x) (as bits n y))
\end{code}
It works as follows:
\begin{codex}
*ISO> nf s 42
43
*ISO> nf p 43
42
\end{codex}
Note that this can be seen as a functor moving operations from a datatype to
another.
We can further generalize it and define the {\em isofunctor} {\tt
borrow\_from}:
\begin{code}
borrow_from lender borrower f = 
  (as borrower lender) . f . (as lender borrower)
\end{code}
It works as follows
\begin{codex}
*ISO> borrow_from bits n s 42
43
*ISO> borrow_from bits n p 43
42
\end{codex}
as well as its 2-argument variant:
\begin{code}
borrow_from2 lender borrower f x y = 
  (as borrower lender) 
  (f (as lender borrower x) (as lender borrower y))
\end{code}
We will now rebuild various arithmetic operations seen as acting on arbitrary
undelimited bitstrings. We start with {\tt db} and {\tt hf} implementing {\em
double} and {\em half} - the last one truncating toward 0:
\begin{code}
db  = p . (0:)
hf = tail . s
\end{code}
Rebuilding hd ({\tt h}), tl ({\tt t}) and cons ({\tt c}) is remarkably easy
- we do not need addition or multiplication yet:
\begin{code}
h :: [N]->[N]
h = h' . p

h' [] = []
h' (1:_) = []
h' (0:xs) = s (h' xs)

t :: [N]->[N]
t = t' . p
t' = hf . t''

t'' (0:xs) = t'' xs
t'' xs = xs

c x ys = s (c' x ys)
c' x xs = c'' x  (db xs)

c'' [] ys = ys
c'' xs ys = 0: c'' (p xs) ys
\end{code}
We can try them out directly on numbers using the {\tt borrow\_from} functor:
\begin{codex}
*ISO> borrow_from2 bits n c 42 2009
17675748928126976
*ISO> borrow_from bits n h 17675748928126976
42
*ISO> borrow_from bits n t 17675748928126976
2009
\end{codex}
Addition can be implemented  by case analysis - i.e. treating the case when
there's a carry-on 0 {\tt sm0} and carry-on 1 {\tt sm1} separately:
\begin{code}
sm xs ys = p (sm0 xs ys)

sm0 [] [] = [0]
sm0 [] (0:ys) = 1:ys
sm0 [] (1:ys) = 0:(s ys)
sm0 (0:xs) [] = 1:xs
sm0 (1:xs) [] = 0:(s xs)
sm0 (0:xs) (0:ys) = 0:(sm0 xs ys)
sm0 (0:xs) (1:ys) = 1:(sm0 xs ys)
sm0 (1:xs) (0:ys) = 1:(sm0 xs ys)
sm0 (1:xs) (1:ys) = 0:(sm1 xs ys)

sm1 xs ys = s (sm0 xs ys)
\end{code}
Multiplication needs to handle {\tt []} (representing {\tt 0}) as a special
case. Otherwise, it just applies the usual algorithm:
\begin{code}
m [] _ = []
m _ [] = []
m xs ys = s (m1 (p xs) (p ys)) where
  m1 [] ys = ys
  m1 (0:xs) ys = 0:(m1 xs ys)
  m1 (1:xs) ys = sm0 ys (0:(m1 xs ys))
\end{code}
Both operations work as expected:
\begin{codex}
*ISO> sm [1,0,1,0] [0,0,1]
[0,0,0,0,0]
*ISO> map (as n bits) [[1,0,1,0],[0,0,1]]
[20,11]
*ISO> 20+11
31
*ISO> as bits n 31
[0,0,0,0,0]
*ISO> borrow_from2 bits n sm 42 13
55
*ISO> borrow_from2 bits n m 5 12
60
\end{codex}
One can see that after shifting the frame of reference to the the bitstring view
of natural numbers we can still work with addition, multiplication etc. as
usual.

\section{Sketch of a fully arithmetized functional programming language}
We have briefly shown that shifting views over data types is quite easy - and we
refer to \cite{arxiv:fISO} for other such isomorphisms between about 60
different data types. We will now focus on programming constructs.
We can start with some simple list operations
\begin{code}
lst n = cons n 0

len 0 = 0
len xs = succ (len (tl xs))
\end{code}
working as follows
\begin{codex}
*ISO> map lst [0..7]
[1,2,4,8,16,32,64,128]
*ISO> as ns n 42
[1,1,1]
*ISO> len 42
3
\end{codex}
Some well known higher order functions (from the Haskell prelude) are next with
names prefixed with {\tt n} to avoid conflicts, when needed:
\begin{code}
nzipWith _ 0 _ = 0
nzipWith _ _ 0 = 0
nzipWith f xs ys = 
  cons (f (hd xs) (hd ys)) 
       (nzipWith f (tl xs) (tl ys))

nzip xs ys = nzipWith cons xs ys

nunzip zs = (xs,ys) where 
  xs = nMap hd zs 
  ys = nMap tl zs
\end{code}
As the following definitions and examples show, more complex data
types like associative lists are easily defined ``inside'' a natural number:
\begin{code}
getAssoc _ 0 = 0
getAssoc k ps = 
  if 0==xy then 0 
  else if k== x then y 
  else getAssoc k (tl ps) where 
    xy=hd ps
    x=hd xy
    y=tl xy
 
addAssoc k v ps=cons (cons k v) ps
\end{code}
\begin{codex}
*ISO> as n ns [0,1,2]
37
*ISO> as n ns [3,4,5]
16648
*ISO> nzip 37 16648
2361183241434889715840
*ISO> getAssoc 2 it
5
*ISO> addAssoc 2 1 (addAssoc 1 2 0)
8392704
*ISO> getAssoc 1 8392704
2
*ISO> getAssoc 2 8392704
1
*ISO> getAssoc 3 8392704
0
\end{codex}
The {\tt fold} family has been shown to be able to express a large class
of interesting other functions as shown in
\cite{DBLP:journals/jfp/Hutton99,DBLP:conf/fpca/MeijerH95}.
Various fold operations are defined along the lines of their Haskell prelude
counterparts:
\begin{code}  
nfoldl _ z 0    =  z
nfoldl f z xs =  nfoldl f (f z (hd xs)) (tl xs)

nfoldr f z 0     =  z
nfoldr f z xs =  f (hd xs) (nfoldr f z (tl xs))

nscanl _ q 0 = lst q
nscanl f q xs = cons q (nscanl f (f q (hd xs)) (tl xs))
\end{code}
They works as follows:
\begin{codex}
*ISO> nfoldl (+) 0  8466
10
*ISO> as ns n (nscanl (+) 0  8466)
[0,1,3,6,10]
\end{codex}
We can now define the equivalent of {\tt reverse}, {\tt map} and {\tt concat}
\begin{code}
rev = nfoldl (flip cons) 0

nMap f ns = nfoldr (\x xs->cons (f x) xs) 0 ns
 
nconcat xss = nfoldr append 0 xss 
 
nconcatMap f xs = nconcat (nMap f xs)
\end{code}
working as follows
\begin{codex}
*ISO> as n ns [1,2,3]
274
*ISO> rev 274
328
*ISO> as ns n 328
[3,2,1]
*ISO> nMap (\x->x^2+1) 274
524548
*ISO> as ns n 524548
[2,5,10]
\end{codex}
Following \cite{mccarthy60} one can build a fully arithmetized theory of
recursive functions together with a Turing-equivalent {\tt
eval} predicate along the lines of \cite{Chaitin75atheory}. Moreover, our
compact bijective encodings are likely to be practical enough for experimenting
with various concepts of algorithmic complexity and randomness \cite{vitanyi}.

\section{A final touch: emulating monadic constructs}
We can now emulate monadic constructs (except that we have $\mathbb{N}$
instead of the parametric type {\tt Monad a}) as follows:
\begin{code}
infixl 1  >>-
m >>- k = nconcatMap k m
nreturn = lst
 
nsequence = foldr mcons (nreturn 0) where 
  mcons p q = p>>- \x -> q >>- \y ->nreturn (cons x y)
\end{code}
The monadic list operation:
\begin{codex}
*ISO>  [0,1] >>= \x->[0,1] >>= \y-> return (x,y)
[(0,0),(0,1),(1,0),(1,1)]
\end{codex}
is now emulated as:
\begin{codex}
*ISO> as n ns [0,1]
5
*ISO> 5 >>- \x-> 5 >>- \y-> nreturn (cons x y)
33058
\end{codex}
which, after shifting to a list view (and then to a list of pairs view) becomes:
\begin{codex}
*ISO> as ns n 33058
[1,3,2,6]
*ISO> map (\x->(hd x,tl x)) [1,3,2,6]
[(0,0),(0,1),(1,0),(1,1)]
\end{codex}
Similarly, {\tt nsequence} emulates Haskell's {\tt sequence} construct as
follows:
\begin{codex}
*ISO> nsequence [1,1,2]
2048
*ISO> map (as ns n) [1,1,2]
[[0],[0],[1]]
*ISO> sequence it
[[0,0,1]]
*ISO> as ns n 2048
[11]
*ISO> as ns n 11
[0,0,1]
\end{codex}

\section{Applications}
Besides their utility as a uniform basis for a general purpose
data conversion library, let us point out
some specific applications of our isomorphisms.

\subsection{Combinatorial generation}
A free combinatorial generation algorithm (providing
a constructive proof of recursive enumerability)
for a given structure is obtained simply through
an isomorphism from $nat$:
\begin{code}
nth thing = as thing n
nths thing = map (nth thing)
stream_of thing = nths thing [0..]
\end{code}
\begin{codex}
*ISO> nth set 42
[1,3,5]

*ISO> nth bits 42
[1,1,0,1,0]

*ISO> take 3 (stream_of hfs)
[H [],H [H []],H [H [H []]]]
\end{codex}

\subsection{Random generation}
Combining {\tt nth} with a random generator for natural numbers
provides free algorithms for random generation of
complex objects of customizable size:
\begin{code}
ran thing seed largest = 
  head (random_gen thing seed largest 1)

random_gen thing seed largest n = genericTake n
  (nths thing (rans seed largest))
  
rans seed largest = 
    randomRs (0,largest) (mkStdGen seed)
\end{code}
For instance
\begin{codex}
*ISO> random_gen set 11 999 3
[[0,2,5],[0,5,9],[0,1,5,6]]
\end{codex}
generates a list of 3 random sets.

For instance
\begin{codex}
*ISO>ran digraph 5 (2^31)
[(1,0),(0,1),(2,1),(1,3),(2,2),(3,2),(4,0),(4,1),
 (5,1),(6,0),(6,1),(7,1),(5,3),(6,2),(6,3)]
 
*ISO> ran hfs 7 30
H [H [],H [H [],H [H []]],H [H [H [H []]]]]
\end{codex}
generate a random digraph and a hereditarily finite set, respectively.

Random generators for various data types are useful for further automating
test generators in tools like QuickCheck 
\cite{DBLP:journals/sigplan/ClaessenH02}
by generating customized random tests.

An interesting other application is generating random problems or programs of a
given type and size.
For instance
\begin{codex}
*ISO> ran sat 8 (2^31)
[[-1],[1,-1],[-1,2],[1,-1,2],[-2],[1,-2],
 [-1,-2],[1,-1,-2],[2,-2],[1,2,-2],[-1,2,-2],
 [3],[1,-1,3],[1,-1,2,3],[1,-2,3],[-1,-2,3],
 [2,-2,3],[1,2,-2,3],[-1,2,-2,3]]
\end{codex}
generates a random SAT-problem.

\subsection{Succinct representations}
Depending on the information theoretical density of
various data representations as well as on the
constant factors involved in various data structures,
significant data compression can be achieved by
choosing an alternate isomorphic representation,
as shown in the following examples:

\begin{codex}
*ISO> as hff hfs (H [H [H []],H [H [],
        H [H []]],H [H [],H [H [H []]]]])
H [H [H []],H [H []],H [H []]]
*ISO> as n hff (H [H [H []],H [H []],H [H []]])
42
*ISO> as ns bits [0,1,0,0,0,0,0,0,0,0,0]
[0,10]
\end{codex}

In particular, mapping to efficient arbitrary length
integer implementations (usually C-based
libraries), can provide more compact
representations or improved performance
for isomorphic higher level 
data representations. Alternatively,
lazy representations as provided by
functional binary numbers
for very large
integers encapsulating results of some
computations might turn out to be more
effective space-wise or time-wise.

Using our groupoid of isomorphisms one can compare representations sharing
a common datatype and make interesting conjectures about their
asymptotic information density, as shown in \cite{arxiv:fISO}.

\section{Related work} \label{related}
The closest reference on encapsulating bijections
as a Haskell data type is \cite{bijarrows} 
and Conal Elliott's composable
bijections module \cite{bijeliot},
where, in a more complex setting,
Arrows \cite{hughes:arrows} are used 
as the underlying abstractions.
While our {\tt Iso} data type is similar
to the {\em Bij} data type in \cite{bijeliot} and
BiArrow concept of \cite{bijarrows},
the techniques for using such isomorphisms
as building blocks of an embedded composition
language centered around encodings
as natural numbers are new.

{\em Ranking} functions can be traced back to G\"{o}del numberings
\cite{Goedel:31,conf/icalp/HartmanisB74} associated to formulae. 
Together with their inverse {\em unranking} functions they are also 
used in combinatorial generation
algorithms
\cite{conf/mfcs/MartinezM03,knuth06draft}.
However the generic view of such transformations as hylomorphisms obtained compositionally
from simpler isomorphisms, as described in this paper,
is new.
Note also that G\"{o}del numberings are typically injective but not {\em
onto} applications, and can only be turned into bijections 
by exhaustive enumeration of their range.
By contrast our ranking/unranking functions are designed to be
``genuinely'' bijective, usually with computational effort 
linear in the size of the data types.

Pairing functions have been used in work on decision problems as early as
\cite{robinson50,robinsons68b}. A
typical use in the foundations of mathematics is
\cite{DBLP:journals/tcs/CegielskiR01}.
An extensive study of various pairing functions and their 
computational properties is presented in 
\cite{DBLP:conf/ipps/Rosenberg02a}.

Natural number encodings of hereditarily finite sets have 
triggered the interest of researchers in fields ranging from 
Axiomatic Set Theory and Foundations of Logic to 
Complexity Theory and Combinatorics
\cite{finitemath,kaye07,DBLP:journals/mlq/Kirby07}.
Contrary to the well known hereditarily finite sets, the concepts of
hereditarily finite multisets and functions as well as their
encodings, are likely to be new, given that our sustained 
search efforts have not lead so far to anything
similar. 

\section{Conclusion} \label{concl}
We have described
encodings for various data types
in a uniform framework
as data type isomorphisms with a groupoid structure.
The framework has been extended
with hylomorphisms providing
generic mechanisms for encoding
hereditarily finite sets, multisets and functions.
In addition, by using pairing/unpairing functions
we have also derived unusually simple encodings for
graphs, digraphs and hypergraphs.

While we have focused on the connected groupoid providing
isomorphisms to/from natural numbers, similar techniques
can be used to organize bijective transformations
in fields ranging from compilation and complexity theory
to data compression and cryptography.

We refer to \cite{arxiv:fISO} for implementations of
a number of other encoders as well as  various  applications.

%\bibliographystyle{INCLUDES/splncs}
\bibliographystyle{plainnat}
\bibliography{INCLUDES/theory,tarau,INCLUDES/proglang,INCLUDES/biblio,INCLUDES/syn}

\end{document}


\begin{comment}
\section{Hereditarily finite representations and inductive graphs}
We will implement here a graph view of various hereditarily finite
representations using Martin Erwig's inductive graph library {\tt fgl}.

\subsection{Representation of hylomorphisms as inductive graphs}
First we transform through {\em deforestation} our {\em unfold} functions
mapping natural numbers to {\tt hfs, hfm} and {\tt hff} trees to generate
a list of edges.

\begin{code}
as_pairg_n = n2es f where f n = [hd n,tl n]

as_ng_n = n2es as_ns_n

as_mg_n = n2es (as ms n)

as_sg_n = n2ps (as set n)

n2es f n = n2edges e f n where  e n m i = (n,m,i)
n2ps f n = n2edges e f n where e n m _ = (n,m,())

n2edges e f n = nub (n2es0 f n) where
  n2es0 _ 0 = []
  n2es0 f n = (zipWith (e n) ms [0..]) ++ 
              (concatMap (n2es0 f) ms) where 
                 ms=f n   
\end{code}

\subsection{The Inductive Graph view}

Then we map the list of edges representations to inductive graphs.
\begin{code}
es2gr :: (Ord a) => [(a, a, t)] -> Gr a t
es2gr es = mkGraph lvs les where (lvs,les)=es2les es

es2les es =  (lvs,les) where 
  vs=to_vertices es
  lvs=zip [0..] vs
  les=to_edges vs es

to_vertices es = sort $ nub $ concatMap f es where
  f (from,to,_label)=[from,to]
  
to_edges vs es = map (f vs) es where
  f vs (x,y,l) = (lookUp x vs,lookUp y vs,l)    
  lookUp n ns = i where Just i=elemIndex n ns
\end{code}


\subsection{A ``visual'' view as inductive graphs}

Finally, we visualize the graphs using graphviz:
\begin{code}
eviz es = gviz (es2gr es)
  
gviz g = writeFile "gr.gv" 
  ((graphviz g "" (0.0,0.0) (2,2) Portrait)++"\n")
 
fig1=eviz . as_pairg_n
fig2=eviz . as_ng_n
fig3=eviz . as_mg_n
fig4=eviz . as_sg_n 
\end{code}

\end{comment}
